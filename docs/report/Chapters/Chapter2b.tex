% Chapter Template

\chapter{Slow-waves} % Main chapter title

\label{Chapter2} % Change X to a consecutive number; for referencing this chapter elsewhere, use \ref{ChapterX}

%----------------------------------------------------------------------------------------
%	SECTION 1
%----------------------------------------------------------------------------------------

\section{Overview of proposed types and functions}
\subsection{Figures}

There will hopefully be many figures in your thesis (that should be placed in the \emph{Figures} folder). The way to insert figures into your thesis is to use a code template like this:
\begin{verbatim}
\begin{figure}
\centering
\includegraphics{Figures/Electron}
\decoRule
\caption[An Electron]{An electron (artist's impression).}
\label{fig:Electron}
\end{figure}
\end{verbatim}
Also look in the source file. Putting this code into the source file produces the picture of the electron that you can see in the figure below.

\begin{figure}[th]
\centering
\includegraphics{Figures/Electron}
\decoRule
\caption[An Electron]{An electron (artist's impression).}
\label{fig:Electron}
\end{figure}

Sometimes figures don't always appear where you write them in the source. The placement depends on how much space there is on the page for the figure. Sometimes there is not enough room to fit a figure directly where it should go (in relation to the text) and so \LaTeX{} puts it at the top of the next page. Positioning figures is the job of \LaTeX{} and so you should only worry about making them look good!

Figures usually should have captions just in case you need to refer to them (such as in Figure~\ref{fig:Electron}). The \verb|\caption| command contains two parts, the first part, inside the square brackets is the title that will appear in the \emph{List of Figures}, and so should be short. The second part in the curly brackets should contain the longer and more descriptive caption text.

The \verb|\decoRule| command is optional and simply puts an aesthetic horizontal line below the image. If you do this for one image, do it for all of them.

\LaTeX{} is capable of using images in pdf, jpg and png format.

There is a large body of literature that addresses slow oscillations in the brain. Slow waves in the delta range (0.5-4 Hz) can been identified in electroencephalograms (EEG) under sleep and anaesthesia. Neurons that switch between up and down states in the respective frequency are present in thalamus indicating that this kind of slow waves is of thalamic origin \parencite[1110]{brown2012control}. More recently neocortical slow oscillations that occur with a frequency below 1Hz have been studied as well. As they do not necessarily occur in regular time intervals they are referred to as neocortical slow waves here (see also section \ref{working_definition}). Both electrophysiological recordings as well as two-photon and wide field fluorescence microscopy have been used to investigate neocortical slow waves \parencite{niethard2018cortical, celotto2020analysis}. Neocortical slow waves with a typical duration of more then one second are the main subject of the empirical analysis presented here.\\
 On one side these low frequency neocortical slow waves and EEG delta waves are recognized as distinct phenomena \parencite[p. 1110]{brown2012control}. On the other hand simultaneous imaging and electrophysiological recordings indicate that both signals relate to each other \parencite{niethard2018cortical}. One interpretation is that neocortical slow oscillations bind together spindles and delta waves \parencite[p. 1110]{brown2012control}. Thereafter one could consider them to be the signature of the same distributed slow wave process. To better understand how different oscillatory patterns in the brain relate to neocortical slow waves is crucial to summarize our knowledge about them. A clear understanding is not only necessary to discern events and measure slow wave properties but it is also important for the cross-validation and interpretation of the results achieved. Hence selected literature on slow waves, sleep and the effect of anaesthetics is briefly reviewed in this chapter \footnote{For an outstanding review with a focus on sleep that includes a discussion about the origins of different kinds of rhythms in the brain see Brown et al. \parencite*{brown2012control}.}. \\
The effects of anaesthesia on the properties and the spiking behavior of individual cells are reported first. The modes of action of anaesthetics such as isoflourane have been studied extensively \parencite{qazzaz2017modulation, moghadam2019comparative, eger1981isoflurane, jenkins1999effects}. It shows that anaesthetics change several properties of neurons which modify the temporal and spatial integration of signals and cause bistable states in distributed neural networks. This can be demonstrated using spiking network models which provide an explanation for the occurrence of neocortical slow waves under anaesthesia that is coherent with named changes of neural properties \parencite{nghiem2018two}. Stronger inhibition of neural activity also exists during sleep. However important characteristics of the dynamics of neocortical slow waves differ between NREM sleep and anaesthesia. Hence it was argued that two different types of slow waves occur during sleep and anaesthesia (see section \ref{effects_of_anaesthesia}). \\
Evidence from several decades of research highlights the role of thalamic circuits plays for the generation of rhythmical activity, the awake state and slow wave sleep \parencite{brown2012control}. While slow wave sleep is arguably promoted by several mechanisms including the cardian cycle and the release of melatonin that alters neural excitability, sleep spindles of thalamic origin are assumed to be the trigger for a transition to deep sleep \parencite[p. 347]{montagna2005fatal}. Besides specific ion channels have been identified in thalamus that presence of which correlates with slow waves in the delta range\parencite[p. 1112]{brown2012control}. Sleep spindles also precede many of the neocortical slow waves observed on the level of single cells using two-photon imaging \parencite{niethard2018cortical}. This contrasts with the assumption that neocortical slow waves form in a spontaneous and independant manner in different regions of cortex if the excitability is reduced. Not all mechanisms of sleep are fully understood. It must be mentioned, however, that sleep is a vegetative state and thalamic nuclei play a crucial role in the generation of neural oscillations including spindles and slow-waves in the theta range (see section \ref{slow_waves_anaesthesia_sleep}).\\
Sleep slow-waves are known to be involved in processes of memory consolidation. The assumed modes of action have been described in the hippocampal neocortical dialog model \parencite{buzsaki1996hippocampo}. Many results support the hypothesis that memory replay occurs during sleep and that it includes both hippocampus and neocortex. One may hypothesize that neocortical slow waves represent the neural signature of an important part of this process. Memory consoldation is arguably the most prominent function of slow waves (see section \ref{slow_waves_and_memory}).\\
As explained above slow-waves differ with respect to several features. This holds both for the duration and frequency at which they occur and the assumed sites of origin. Neocortical slow waves can emerge from various regions. EEG studies in humans indicate that prefrontal-orbitofrontal regions act as the preferred location of early above average activity \parencite[p. 1110]{brown2012control}. First steps to distinguish slow-waves in a data driven approach were carried out by Bernardi et al. \parencite*{bernardi2018local}. Two types of sleep slow-waves may be distinguished in EEG. This distinction is based on a metric that includes a shape parameter and a term that indicates whether slow-waves are rather local or widespread. The distibution of these types of slow-waves differs during different stages of deep sleep. It highlights the importance of means to discern slow-waves based on their shape and spatial pattern of activation. A working definition that allows to detect and separate slow-waves is derived from the developed understanding of slow-waves (see section \ref{working_definition}).

%----------------------------------------------------------------------------------------
%	SECTION 2
%----------------------------------------------------------------------------------------

\section{Effects of anaesthesia on single cells and neural networks}
\label{effects_of_anaesthesia}
Anaesthetics alter the spiking behaviour of cortical neurons. Under deep anaesthesia the activity of cells is highly reduced and quiescence can be observed for most units. However, the modes of action that cause the respective changes of neural properties differ between anaesthetic agents. In addition, the effects can depend on the exact concentration of the drug in use and low dosages may, in some cases, even have opposing effects. As isoflourane was used to aquire the dataset at hand its effects on individual cells is shortly summarized highlighting important differences to other anaesthetics when necessary. It is known that alterations of the properties of individual cells change the dynamics of neural interaction on a population level. By adjusting the properties of cells in simple models of neural networks bistable states can be reproduced that resemble slow-waves under anaesthesia.\\
The exact pharmacological mechanisms of action of inhalant anaesthetics such as isoflourane remain uncertain (Miller et al. 2020). Effects on several ion channels have been reported including both chemically gated CL- channels (GABA receptors and glycine receptors) and K+ channels (Glutamate receptors). For example isoflourane is known to reduce the hyperpotentiation that results from CL- influx as a consequence of in vitro GABA administration (Jenkins 1999). Hence it can be assumed to have excitatory effects in the nervous system as it decreases the inhibition due to GABA. However isoflourane also has inhibitory effects as it supresses K+ channel currents resulting in smaller electrically triggered peak amplitudes of action potentials (Buljubasic 1992). Besides a potentiation of glycine receptors is assumed alongside other neurochemical mechanisms that alter the excitabilty of neurons (National Center for Biotechnology 2021). In vivo studies indicate that the net-effect of isoflourane appears to be inhibitory for all relevant dosages. In this respect isoflourane contrasts with other anaesthetics (including e.g. halothane and ketamine) that show concentration dependance (Mogdahan 2019). It shall be noted however that differences in the density of different types of receptors exist in different areas of the brain. While isoflourane can be assumed to decrease the excitability of neurons and inhibit neural signaling at all concentrations this effect could differ between brain areas.\\
Single cell recordings reveal decisive effects of Isoflourane on the properties and the spontaneous bahavior of neurons. Moghadam et. al (2019) performed a comparative study of systemic and volatile general anesthetics in single cell cultures and the isolated brain of lymnaea stagnalis. The substances tested include the volatile agents sodium pentobarbital, sodium thiopentone, ketamine on one side as well as halothane, enflurane and isoflurane on the other. It was found that isoflourane causes a gradual decline of both amplitude and frequency of spontaneous action potentials. Differences between the six types of neurons studied were found to be marginal. Upon stimulation neurons remained silent at all examined concentrations of isoflourane. In contrast a gradual decline of evoked action potentials showed for inceasing levels of enflourane. Using either halothane or barbiturates the authors were also able to produce bistable states in vitro during which periods of rapid spiking and quiesence alternate spontaneously.\\
Besides named alternations in spiking, effects on subthreshold properties of neurons have been identified. Increasing levels of isofourane can cause a decrease in the membrane time constant i.e. the duration between stumulus onset and 63\% potential change of the cell membrane. Under normal conditions it takes longer for the neuron to reach maximal voltage as compared to anaesthesia with isoflourane. This is reflected in the estimated membrane capacitance that analogously decreases. The membrane capacitance is especially interesting because of the role it plays in the integration of electrical inputs (Golowasch \& Nadim 2014). The most likely explanation for the apperent reduction is however an increase in the leakage current. This is because it is (1) rather unplausible for the capcitance of the constant size membrane to change significantly and (2) because of the abovementioned interaction between membrane proteins and anaesthetics (Quazzaz \& Winlow 2017). Neurons act as temporal integrators and fire if the combined voltage of input spikes that occur simultanously or in short succession exceeds threshold. If the leakage current is stronger the timing of input spikes becomes more critical as the membrane potential goes back to baseline more quickly which may prohibit charge accumulation. Hence it can be hypothesized that anaesthetics affect temporal integration.  \footnote{ Note that other anaesthetics were found to increase the membrane capacitance}\\
Arguably the abovementioned effects explain important features of population level dynamics. Marshall (1981) systematically studied the EEG patterns in the awake state and during anaesthesia with isoflourane for five different dosages at up to 2.9\% in humans. For very light anaesthesia (iso = .56\%) low voltage fast activity can be observed. At a light surgical level (isp = .56\% and 1.78\%) slow oscillations are present that change from more regular to irregular patterns and alternating patterns with high amplitude oscillations at a moderate surgical level (iso = 2.2\%). For deep anaesthesia only occasional low voltage activity shows (iso = 2.9\%). Isoflourane administration leads to a gradual shift from a stable awake state to bistable states and finally deep anaesthesia where quiesence dominates.\\
The effects of anesthesia presented here provide the background for a mechanistic interpretation of slow-wave activity. The overall dynamics that arise from changes of neural properties can be modeled using a simple model that employs Adaptive Exponential Integrate And Fire cells as shown by Nghiem et al. (2018): Simulations of a network that consists of 80\% excitatory cells and 20\% fast spiking inhibitory neurons produce sequences of continuous activity (up-states) and widespread absence of action potentials (down-states). A shift from down-states to and up state can be triggered by background noise while "spike-frequency adaptation on excitatory cells produces a self-inhibition that, destabilizing the up state, causes a reset to the down state" (Nghiem et. al 2018, p. 2). Spiking network models can reproduce alternating sequences of up states and down states, an important characteristic of population level activity under anaesthesia. \\
While simple mechanistic models explain some of the features of anaesthesia on a population level they represent a coarse generalization. It was highlighted that the exact mechanisms of action of anaesthetics remain uncertain. Generalizing over different agents and dosages is not necessarily justified and the dynamic of neural signalling might be affected by differences in the distribution of receptors for different neurotransmitters. Besides the complex anatomy of the brain including the various pathways that connect different regions uni- or bidirectionally is typically not reflected in simple mechanistic models. More precise measurements of the pathways and dynamics of neural signal transduction during slow-wave anaesthesia are necessary to understand bistable states of the brain. This is especially important also because it was recently shown that slow-wave sleep differs significantly from anaesthesia.

%-----------------------------------
%	SECTION 3
%-----------------------------------
\section{Slow-waves in anaesthesia and deep sleep}
\label{slow_waves_anaesthesia_sleep}
Recently it was argued that the dynamics of slow-wave sleep and slow-wave anaesthesia differ substantially \parencite{nghiem2018two}. Jercog et al. \parencite*{jercog2017up} found that the length of up states and the subsequent down state is correlated during urethane induced anaesthesia in rats for clearly synchronized periods where high-amplitude, slow fluctuations are present in local field potentials. Coefficients indicate a very weak relationship (r = .2). However, the correlation was found to be consistently positive across experiments whereas the correlation with later down-state periods (time-lag > 2) is close to zero \parencite{jercog2017up}. This pattern in the dynamic of neural firing can be reproduced using simple mechanistic models (see section \ref{effects_of_anaesthesia }). During human non-REM sleep, however, no such relationship holds. As a tendency, long down states are followed by short up states instead as indicated by a very weak but significant negative correlation (r = -.04). A negative peak in the temporal cross correlation exists around zero\parencite{nghiem2018two}. This gave rise to the argument that sleep-slow waves are fundamentally different from their counterpart during anaesthesia \parencite{nghiem2018two}. Hence the mechanisms that enable slow wave sleep are more closely examined in this section. \\

%-----------------------------------
%	SECTION 3.1
%-----------------------------------

\subsection{Three systems for sleep regulation}
It is long known that the anatomical substrate of sleep is a distributed system rather than a single region in the brain\parencite{akert1965anatomical}. Mammalian sleep represents a physiological process that is arguably regulated by the interaction between various control mechanisms. In general, one may categorize these processes into three classes each of which affects the activity of cortical neurons directly or via intermediate events. First there are processes that decrease neural excitability rather unpacifically by a release or an accumulation of disperse chemical messengers. Second processes exist that incorporate electrical signalling and the neurotransmitter systems that correspond to the ascending reticular activating system. Third there are processes that lead to oscillations in the firing rate of distinct neural populations. The latter can include spike rates that follow the circadian rhythm and alter the activity of both other types of processes. However oscillatory activity also occurs in the delta range. As it is not possible to discuss all features of sleep here, important examples are presented that fall under named categories. The aim is to characterize the circumstances under which sleep slow waves occur.\\
Melatonin is the messenger of a system for sleep regulation that falls under the first category. In healthy subjects, melatonin concentrations alternate according to the cardiac cycle \parencite{montagna2005fatal}. Melatonin can pass the blood brain barrier and can hence diffuse to the central nervous system where it accumulates \parencite{aulinas2019physiology}. It acts as a neurohormone and has an inhibitory effect on the excitability of cortical neurons via different modes of action. Melatonin receptors in the neural membrane have been identified. Besides it was found that melatonin interacts with voltage-sensitive Ca2+ channel and inhibits the release of neurotransmitters as well as synaptic transmission \parencite{choi2014melatonin}. As it decreases the neural excitability of cortical neurons it shares some of the effects of mild anaesthesia. Melatonin reduces the excitability of disperse populations of neurons in the central nervous system promoting drowsiness and sleep. \\
The suprachiasmatic nuclei (SCN) represent an example for the third of the abovementioned categories. It interacts with the melatonin system. Melatonin is produced by pinealocytes that compose 95\% of the cells in the pineal gland \parencite{aulinas2019physiology}. The pineal gland resides outside the blood brain barrier and extends hypothalamus ventrally and represents an interface between neural signalling end the endocrine system. It is innervated by centres in the brain stem. Most importantly it is known to receive signals which are related to retinal activity from the SCN. The SCN is capable not only of creating a light-dependant circadian rhythm, but also of maintaining an entrained rhythm \parencite{koella1984organization}. The latter could be demonstrated in electrophysiological recordings in vitro. Isolated SCN show firing rates that alternate in a 24 hour pattern \parencite{de2011melatonergic}. Arguably this explains changes of melatonin release that follows the circadian cycle.\\
While melatonin is evidently part of sleep regulation the transition between the awake state and slow wave sleep is arguably initiated by other mechanisms. Mammalian sleep can be triggered by electrical stimulation of the thalamus. First experiments were carried out using cats while later experiments confirmed that the effect exists for dogs as well. A stimulation of the intralaminar thalamus of cats with a duration of 30-60 seconds and pulses at 4-8Hz leads to a transition to sleep. This transition occurs in different stages. Sleep spindles occur while the animal is still awake. After several minutes a full transition to slow-wave sleep can be observed. Similar to intralaminar thalamus the electrical stimulation of the raphe nuclei induces sedation and sleep in rats. The raphe nuclei are brain stem areas that represent of origin of the ascending serotonergic pathways and hence play an important role in the arousal system. The mechanisms that lead to a transition to sleep include an inhibition of the ascending arousal system and hence decrease the excitatory input to cortical neurons in the sense of the second of the abovementioned systems.\\
It shows that the mechanisms that regulate sleep decrease the excitability of cortical neurons under the contribution of chemical messengers including melatonin and an inhibition of the ascending arousal system. Subcortical nuclei can produce oscillatory patterns that reflect the circadian cycle whereas thalamus plays a key role in the transition from an awake state to deep sleep. Deep sleep represents a bistable and is characterized by a switch between up an down states of cortical activity. Thalamic nuclei are known to produce sleep spindles which are considered a switch for the transition between light an deep sleep \parencite{montagna2005fatal}.

%-----------------------------------
%	SECTION 3.2
%-----------------------------------

\\subsection{The ascending reticular activation system and neural oscillators}
Both the transition to the bistable state of slow wave sleep and fluctuations in the delta include ascending signals that originate from subcortical areas. Several centres that generate different oscillatory patterns have been identified. These patterns include thalamocortical sleep spindles, hippocampal short wave ripples and slow waves in the delta range that are of thalamic origin.\\
Early lesion studies indicated that brainstem nuclei are required for both normal awake activity and REM sleep. In animals with a Cerveau isolé preparation only slow waves can be observed. In named preparation ascending fibers between pons and midbrain are dissected. Effectively large parts of the reticular formation and several nuclei that use specific neurotransmitters are disconnected from midbrain, thalamus and cortex. In contrast an encéphale isolé preparation in which the brain and the spine are separated does not have named effects but normal sleep patterns can be observed. This highlights the importance of brain stem for the regulation of sleep and alertness. The nuclei mentioned above including serotoninergic neurons in the raphe nucleus, norepinephrine neurons in locus coeruleus and glutamate neurons in the pedunculopontine nucleus - the most important excitatory neurotransmitters. These nuclei are known as the source of the ascending reticular activation system.\\
The circumstance that slow waves were present in the Cerveau isolé preparation indicated that slow waves are initiated in brain regions above pons. Thalamus is considered a relay for ascending signals in the awake state. However it also contains nuclei that act as neural oscillators which is especially under certain conditions, most notably for sleep. "During these states, the behavior of thalamic cells is characterized by long-lasting hyperpolarizations and phasic burst discharges recurring rhythmically" (Steriade 1984, p. 21). Chirugical removal of thalamus in cats was found to lead to a decrease of the amound of non-REM sleep from 38\% to 11.8\%. In contrast diencephalic cats still showed sleep spindles (after Montagna 2005). Further support comes from findings from fatal familial insomnia which indicate that thalamic atrophy is "associated with lack of sleep spindles and delta sleep implicate the thalamus in the origin of slow wave sleep"(Montagna 2005).
Sleep slow waves im the theta range are generated in distinct nuclei of thalamus. Rostral midline thalamus (RMT) contains neurons which show alternating patterns of spiking activity that oscillate in the theta range. Another thalamic nucleus is arguably the source of sleep spindles i.e. sequences of synchronized high frequency bursts. They are considered to be a switch for the transition between light an deep sleep and precede many neocortical slow oscillations.\\
Neocortical oscillations occur spontaneously during deep sleep and deep anaesthesia with distinct frequencies of below 1 Hz. The hypothesis that neocortex contains networks that act as a neural oscillator themselves has been confirmed both empirically and by means of simulation (see also section \ref(effects_of_anaesthesia)). Dissection. Name example for empirical finding. Figure 1 illustrates the location of oscillatory nuclei and shows the ventral and dorsal pathway of the ascending reticular activation system. While ... interconnected\\
Proposed function of Medial nucleus and ventrolateral nucleus for sleep initiation. production and autonomic balance.\\
Interaction effects between signals that arise from thalamus and those in cortex have been identified. Thalamus is evidently the source of sleep spindles and theta waves, the most important patterns of slow wave sleep. However it shall be mentioned that thalamus is strongly interconnected with cortex. Electrical stimulation of the reticular core of thalamus was found to cause an increased theta power in hippocampus. In contrast rhythmical stimulation of cortical area 7 evokes periodic oscillations in thalamic neurons that sustain over extended periods of time after stimulation offset. The resulting thalamic rhythm can in return enhance synchronization (Steriade 1984). This finding indicates not only that entrained rhythms exist in thalamus but also that oscillations in thalamus and both archiocortex and neocortex strongly relate to each other.\\

%-----------------------------------
%	SECTION 3.3
%-----------------------------------
\subsection{Neocortical slow waves during sleep and anaesthesia}
Thalamus gateway to consciousness. In awake. Relay. However it was found that the thalamus can also act as neural oscillator. In sleep. "During these states, the behavior of thalamic cells is characterized by long-lasting hyperpolarizations and phasic burst discharges recurring rhythmically" (Steriade 1984, p. 21). Thalamic -> Athalamic cat: non-REM sleep 38\% -> 11.8\% of total observation time. Diencephalic cats (ablation of the entire neocortex and striatum): Still Sleep spindles (after Montagna 2005).\\
•	lesioning the reticular nucleus also decreased delta waves and led the animals to sudden death.36 \\
 Preferential thalamo-olivary degeneration. "paramedian thalamus acts as ‘final common pathway’"??\\

Spindles under anaesthesia with barbiturates.

This gives rise to the question: What is sleep and how is it different from anaesthesia? what is sleep.
Explains dynamics during anaesthesia not sleep. Sleep.... hippocampus.
Long up states --> long down states. during anaesthesia.

Functions of sleep. short. Ending with the role of memory formation. Question wheather it is different is unclear.
The circumstance that it is debated whether sleep-slow waves and slow-waves under anaesthesia are comparable highlights the importance of an approach that allows to automatically characterize these patterns of neural signals.


Methodological triangulation.

%-----------------------------------
%	Section 4
%-----------------------------------

\section{Sleep slow-waves and memory consolidation}
\label{slow_waves_and_memory}

According to the hippocampal-neocortical dialogue model of slow-waves, the interaction of hippocampal sharp wave ripples and neocortical slow-waves fosters memory consolidation (Buzsáki 1989, Walker et al. 2009): Recently it was shown that generating neocortical slow-waves in prefrontal networks such that they are coupled to the occurrence of short wave ripples in the hippocampus increases the performance of rats in a memory task (Maingret 2018). Coherently, a correlation of slow-wave activity and the brain-derived neurotrophic factor was found for humans (Duncan, 2013). Spontaneously occurring slow-waves arguably play a similar role for memory. They predominantly occur phase locked to sleep spindles (Demanuele et. al. 2017). However, the exact mechanisms that orchestrate this synchrony are unknown (Sanda et. al 2020). Moreover, it was found that at least two different types of slow-waves exist that potentially relate to distinct synchronization processes (Bernardi 2018). Because of its high spatial resolution fluorescence microscopy can provide more fine-grained distinctions and may potentially reveal trajectories of neural signal transduction during slow wave anesthesia. This highlights the importance of methods that allow to capture the variance of temporospatial patterns of neural slow-waves.

%-----------------------------------
%	Section 5
%-----------------------------------

\section{A working definiton}
\label{working_definition}
A distinction between slower and shorter waves has already been made in the pioneering electroencephalographic study almost a century ago \parencite[p. 550]{berger1929}. Shortly later the term delta wave was coined to describe oscillations in the recorded voltage in the respective frequency band of around 1 to 4 Hz. The term slow-wave is typically also used to describe oscillatory patterns in EEGs. "Slow-wave activity (SWA) is defined as the EEG power in the slow wave frequency band" \parencite[p. 1]{furrer2019sleep}. This frequency band relates to the delta band. Hence the term delta wave which is oftenly used in EEG research can be considered a synonym for slow waves. However, when speaking of slow-waves one oftenly referrs to slow wave sleep or anaesthesia where delta waves occur largely independant from higher frequency components.\\
As explained above slow waves are however more then an oscillation in the scalp potential relative to a reference electrode. The term is also used in relation to the neural phenomenom, its functions and its signatures that can be measured using various methods. Slow-waves are short periods of above baseline activity that occur in bistable states such as deep sleep and anaesthesia which are characterized by reduced neural excitabilty. Slow-waves are dynamic patterns of recurrent acticity of large potentially disctinct populations of neurons. They can differ with respect to the pathway of neural flow and the set of recruited cells. Sleep slow waves represent an important mechanism for memory consolidation. Evidence suggests that memory replay occurs both for sparse hippocampal codes and the associated dense cortical manifestations of the previously encoded learning experience. If thalamic short wave rippels correlate with cortical up-states one may speak of a thalamocortical slow wave event. Analogously one could form subcategorizations for slow-waves that incorporate deviant sets of temporally connected neurons. One may hypothesize that slow waves show neural associations in temporal isolation.  \\
If slow-waves are understood as short periods of above baseline activity they can hence incorporate multiple peaks or slow oscillations in the strict sense. This is especially important for the detection and separation of events. Instead of relying on local minima alone this definition requires to detect in addition wheather activity went back to baseline. While conceptualizing slow waves as slow oscillations in the strict sense this approach chosen. It bears potentials of revealing the temporal dynamics of slow waves that incorporate several stages.
