% Chapter Template

\chapter{Discussion} % Main chapter title
%----------------------------------------------------------------------------------------
%	SECTION 1
%----------------------------------------------------------------------------------------
In summary, the combination of Optical Flow, Helmholtz Decomposition and Autoencoders allows to quantify important features of neocortical slow waves and reveals a complex topology of events in latent space. The approach has proven to be viable for the suggested event related analysis. However, improvements regarding the estimate of the df/f signal are necessary and the origin of a breathing related oscillations must be more closely examined. An event related analysis of neocortical slow waves and the use of VAEs on sparse features helps to distinguish different kinds of neocortical slow waves.\\
As the literature review showed, neocortical slow waves can originate in cortex spontaneously. However, they can not only be assumed to play an important role for memory but also to interact with other rhythms in the brain. Neocortical slow waves can be observed during bistable states of anaesthesia and deep sleep where both periods of recurrent activity and quiescence can be observed. Isoflurane induces a decrease in the junctional conductivity. Down states are arguably caused by Ca2+ accumulation that triggers Ca2+ gated K+ channels and leads to a population level refractory period \parencite{neske2016slow}. Neocortical slow waves can be triggered by subcortical stimulation \parencite{stroh2013making}. They typically incorporate only a subset neurons in cortex with strong interprojections. Arguably this explains why distinct areas of recurrent activity and decisive patterns of flow can be observed during different events.\\
Optical Flow and Helmholtz decomposition can help to measure pathways of flow and the centers of recurrent activity such that different types of events can be distinguished. It can hence be used as an alternative to approaches that detect the movement of wavefronts. Gradient based methods for Dense Optical Flow yield dense vector fields of displacement vectors for levelset boundaries (see section \ref{chapter_optical_flow}). This allow to measure more complex patterns which is especially important for neocortical slow waves with extended up states that exhibit a complex dynamic. These patterns of flow cannot be sufficiently described by moving wavefronts. Focal areas of brightness increase reflect areas of functional connectivity, manifest as sources in the vector fields and can be distinguished from patterns in the flow using Helmholtz decomposition (see section \ref{section_helmholtz_decomposition}). \\
Variational Autoencoders with two dimensional latent layers appear as a good choice for dimension reduction because the latent space distributions could easily be interpreted. Manifolds of reconstructions can be computed for points on an evenly spaced grid using the decoder (see section \ref{section_autoencoders}). This allows to reduce the dimensionality of the data and present it in a way that is intuitively understandable by the reader. Of the three models implemented here, the mixed input Autoencoder allows to distinguish events in the most fine-grained manner.\\
In general, it shows that the amplitude of events decreases with the dosage of anaesthesia. The flow-component of the vector fields retrieved with the Horn and Schunck method reveal a complex dynamic of events. Different types of slow waves can be distinguished in the latent space distributions that Autoencoders reveal. Large amplitude waves either refer to medial axis events or events that incorporate frontal regions as sources.\\
Medial axis events typically include RSD and secondary motor areas, in some cases, show decisive upwards flow. Occipito-frontal flow occurs especially in an early stage. In later stages of some events medial lateral flow can dominate and wide regions of the frontal lobe can act as centers of recurrent activity that include primary motor areas. As the mixed input Autoencoder shows, slow waves with sources close to the medial axis exhibit a trend towards upwards flow. However, the strongest flow can also occur in horizontal direction or in all four main directions. This arguably reflects the described spread from medial to lateral or a complex dynamic of flow patterns respectively. Medial axis events typically have a single peak with a steep rise and decline. The circumstance that RSD is involved leaves room for the hypothesis that also the neighbouring hippocampus might play a role. Cingulate fibers originate from entorhinal cortex which is also closely connected \parencite{milczarek2018spatial}. Respective trajectories in human EEG have been reported \parencite{murphy2009source}. The explanation that these events represent hippocampo neocotical slow waves is, however, only one possible explanation. Bottom up events could also be triggered by visual input to occipital cortex. Arguably, further research is necessary to explain the origin of this kind of neocortical slow waves.\\
The second type of large amplitude waves incorporates frontal to fronto-parietal sources. In some cases, downwards flow is stronger then flow in other directions. Typically however, flow in all main directions occurs. This arguably reflects the complex dynamic of this type of slow waves that, in many cases, incorporates multiple peaks. Frontal areas have also been reported to be the main sources of slow waves in humans \parencite[p. 1110]{brown2012control}. Neocortical slow waves with multiple peaks that reflect extended upstates occur predominantly during light anaesthesia.\\
Low amplitude events occur at higher levels of isoflurane. One type of low amplitude wave shows mainly frontal sources and has a second peak that is correlated with the hemodynamic breathing signal. This can be interpreted as evidence for neocortical slow waves that are synchronized with breathing. It is unclear, however, whether the breathing synchronized peaks reflect a hemodynamic effect or also relate to neural oscillations that occur simultaneously. The analysis of low amplitude slow waves also shows events that have sources in the barrel fields. As they process signals from the whiskers it appears possible that breathing coupled sensory input exist. Arguably further research is necessary to study the interaction between neocortical slow waves and breathing. As the effect of breathing correlated components is negligible at anaesthesia levels with a better signal to noise ratio it is advisable to employ the proposed approach for the respective conditions.\\
The signal to noise ratio at high levels of anaesthesia represents a challenge. In addition, a difficulty arises from the circumstance that the standard deviation of the detrended hemodynamic signal increases for higher levels of isoflurane. Moreover, the GCaMP signal decreases in strength in these conditions. This means that the experimental conditions are systematically confounded with a potential error signal.\\
Additional refinements in the preprocessing might be necessary to further reduce possible artifacts due to hemodynamic effects. Subtracting a regression fit has proven difficult as the hemodynamic signal for a given pixel showed to be only marginally correlated with the GCaMP signal. Potentially this is the case because a breathing synchronized neural oscillation coincides with the peaks in the hemodynamic signal. Whether this is the case could however not be finally answered. Named challanges limit the interpretability of the results achieved for high levels of anaesthesia. Using the approach to characterize slow waves for conditions where the signal to noise ratio is high can help to avoid potential problems.\\
Other improvements can be made with respect to the computation of the df/f signal. Instead of computing a contrast to the temporal mean per event one may compute the contrast to the temporal mean of down state frames. This strategy worked well for high signal to noise conditions with decisive down states. The way to compute the df/f that was used here was found to yield highly similar results. Although refinements in the preprocessing might be necessary, the approach is considered a valid method to analyze slow waves in flouroscence recordings. It allows to study neocortical slow waves using widefield fluorescence imaging.\\
Optical Flow, Helmholtz Decomposition and Autoencoders can be used to characterize slow waves. This bears multiple potentials for the analysis of neocortical events that occur under anaesthesia. As widefield fluorescence microscopy is challenged by the hemodynamic autofluorescence it is advisable to select stages of anaesthesia where neocortical slow wave occur with a good signal to noise ratio and to employ methodological triangulation: The use of fluorescence microscopy in combination with electrophysiological recordings has previously been shown to help in the interpretation of the results. The approach shown here could in principle even be used with ECoG or EEG. In future Optical Flow, Helmholtz Decomposition and Autoencoders can be used to characterize neocortical slow waves and put their properties into perspective with effects of anaesthetics. However the approach could also be used to study other contingencies as well as sleep slow waves. As sleep slow waves are thought to foster memory consolidation one may put fourth the hypothesis that different types of events exist which reflect the interaction between hippocampal sharp waves and neocortical slow waves. Optical Flow, Helmholtz Decomposition and Autoencoders can be used to study neocortical slow waves during sleep and anaesthesia using widefield flouroscence imaging or electrophysiological methods.
