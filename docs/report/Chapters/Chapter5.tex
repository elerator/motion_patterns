% Chapter Template

\chapter{Discussion} % Main chapter title

%----------------------------------------------------------------------------------------
%	SECTION 1
%----------------------------------------------------------------------------------------
Summarize your key findings
In summary the combination of Optical Flow, Helmholtz Decomposition and Autoencoders reveals a complex topology of different types of neocortical slow waves. It has proven to be a viable approach for the suggested event related analysis of neocortical slow waves. As the literature review showed, neocortical slow waves can originate in cortex spontaneously, however, they can not only be assumed to play an important role for memory but interact with other rythms in the brain. Slow waves can be triggered by subcortical stimulation. They typically incorporate only a subset of strongly interconnected neurons in cortex which results in distinct areas of recurrent activity and patterns of flow. Optical Flow and Helmholtz decomposition can help to measure pathways of flow, centers of recurrent activity and help to distinguish different types of events. Although refinements in the preprocessing might be necessary to further reduce potential artifacts due to hemodynamic effects of breathing the approach can be considered a valid method to study neocortical slow waves using widefield flourescence imaging.\\
In general it shows that the amplitude of events decreases with the dosage of anaesthesia. The flow-component of the vector fields retrieved with the Horn-Schunck method reveal a complex dynamic of events. Different types of slow waves can be distinguished using autoencoders. Large amplitude waves either refer to medial axis events or events that incorporate frontal regions as sources.\\
 Medial axis events typically include RSD and in some cases show decisive upwards flow, especilly in an early stage. In later stages medial lateral flow can dominate and frontal regions act as centers of recurrent activity. As the mixed input autoencoder shows slow waves with sources close to the medial axis exhibit a trend towards upwards flow in average. However, the strongest flow can also occur in horizontal direction or in all four main directions for many events. This arguably reflects the described spread from medial to lateral or a complex dynamic of flow patterns. Medial axis events typically have a single peak with a steep rise and decline.\\
The second type of large amplitude waves incorporates frontal to fronto-parietal sources. In some cases downwards flow is strong, typically however, flow in all main directions occurs. Arguably this reflects that this kind of neocortical slow wave
The method represents a valid tool to analyze neocortical slow waves.\\
Low amplitude waves occur at higher levels of isoflurane. One type of low amplitude wave shows a second peak that is correlated with the hemodynamic breathing signal. This potentially indicates that certain neocortical slow waves are synchronized with a breathing signal. It is unclear, however, wheather the breathing synchronized peaks reflect a hemodynamic effect or relate to neural oscillations that occur simultaneously. This implies that further research is necessary to study the interaction between neocortical slow waves and the breathing signal. As the effect of breathing coupled slow waves could easily be removed for anaesthesia levels with a better signal to noise ratio it is advisable to employ the proposed approach for named conditions first.\\
Helmholtz decomposition helps to distinguish focal areas of recurrent activity from global patterns of flow. It can hence be used as an alternative to approaches that detect the movement of wavefronts. Optical flow allows to measure more complex patterns which is especially important for neocortical slow waves with extended up states that exhibit a complex dynamic that is not suffiecently described by moving wavefronts.\\
The autoencoder with mixed input argubly allows to distingush events in the most finegrained manner. Variational autoencoders with two dimensional latent layers appear as a good choice especially because the latent space distributions could easily be interpreted using the manifolds of reconstructions that can be computed for points on an evenly spaced grid using the decoder. This allows to reduce the dimensionality of the data and present it in a way that is intuitively understandable by the reader.\\
The signal to noise ratio at high levels of anaesthesia represents a challange. In addition, a difficulty arises from the circumstance that the standard deviation of the detrended hemodynamic signal increases for higher levels of isoflurane. Moreover the GCaMP signal decreases in strength in these conditions. This means that the experimental conditions are systematically confounded with and error signal. As discussed is unclear what low amplitude oscillations that correlate with breathing relate to. Potentially, additional refinements in the preprocessing might be necessary to further reduce potential artifacts due to hemodynamic effects. To avoid potential effects of spatially nonuniform activity the possibility to compute the contrast to a downstate image for each event should also be assessed. Named problems limits the interpretability of the results achieved for high levels of anaesthesia. Nonetheless, the suggested approach represents a valid method to study slow waves: An event related analysis of neocortical slow waves and the use of variational autoencoders on features retrieved with Optical Flow and Helmholtz decomposition helps to distinguish different events.\\
Optical Flow, Helmholtz Decompositon and autoencoders can be used to characterize slow waves. This bears multiple potentials for the analysis of neocortical slow waves that occur during sleep but also under anaesthesia. If sleep slow waves foster memory consolidation one may hypothesize that different types of events exist during sleep that reflect the interaction between hippocampal sharp waves and neocortical slow waves. As widefield flouroscence microscopy is challanged by the hemodynamic autoflourescence it is advisable to select stages of anaesthesia where neocortical slow wave occcur with a good signal to nosie ratio and to employ methodological triangulation. The use of flouroscence microscopy in combination with electrophysiological recordings has previously been shown to help in the interpretation of the results. In future Optical Flow Helmholtz Decomposition and Autoencoders could potentially be used to characterize neocortical slow waves and put their properties into perspective with animal behavior or leaning related signals in hippocampus. 
