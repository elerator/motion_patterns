% Chapter 1

\chapter{Introduction} % Main chapter title

\label{Chapter1} % For referencing the chapter elsewhere, use \ref{Chapter1}
%----------------------------------------------------------------------------------------
% Define some commands to keep the formatting separated from the content
\newcommand{\keyword}[1]{\textbf{#1}}
\newcommand{\tabhead}[1]{\textbf{#1}}
\newcommand{\code}[1]{\texttt{#1}}
\newcommand{\file}[1]{\texttt{\bfseries#1}}
\newcommand{\option}[1]{\texttt{\itshape#1}}
%----------------------------------------------------------------------------------------
Anaesthesia inhibits neural activity and alters the signals in the brain. Depending on the dosage of the anaesthetic in use different patterns can be observed \parencite{eger1981isoflurane}. At intermediate concentrations many agents cause propagating low-frequency waves that occur in close succession and manifest as slow oscillations in EEG \parencite{steriade1993novel, eger1981isoflurane}. Similar patterns can be observed during deep sleep. Neocortical slow waves differ in shape and size and with respect to specific temporo-spatial features: Most importantly the source location of recurrent activation, the direction and speed of spread and different sets of functionally connected regions can be observed \parencite{brown2012control}. These features can be measured using modern calcium imaging techniques that rely on GCaMP flourosence.\\
The development of new experimental technologies to capture calcium fluorescence with an increasing temporal resolution of up to 100Hz offers new opportunities for the study of the neural circuitry and dynamic state changes of the brain under anaesthesia. At the same time, they also impose new challenges that result from the complexity of the recorded data. These challenges must be addressed by a well-controlled experiment or suitable approaches for data analysis.\\
Strict experimental control simplifies data processing. Under certain experimental conditions, high amplitude slow waves with a good signal to noise ratio and contrast to the hemodynamic autoflourescence occur in a regular pattern. For ketamine induced anaesthesia (100mg/kg), regular patterns of neocortical slow waves with highly similar shapes, durations and amplitudes occur. This simplifies the separation of events and the subsequent analysis \parencite{celotto2020analysis}. However, such strict control means that the variance of the captured slow waves is highly reduced.\\
More general findings regarding the structure of neocortial slow waves are only possible in less controlled settings. This means, however, that suitable methods for data processing and analysis must be established that are both robust and suitable to capture the dynamics of neocortical slow waves of different shapes and signal strengths.\\
Hence a new approach was developed that allows to distinguish and characterize neocortical slow waves of various shapes and sizes. It reveals a complex topology of events and shows how different levels of isoflurane change the way signals travel in the brain. The development was accompanied by two major questions: (1) What are slow waves and (2) how can their temporospatial properties be measured and characterized. To answer the earlier question and provide a working definition for neural slow waves that allows to distinguish different events, relevant slow wave literature was systematically reviewed. To tackle the latter question, it was investigated how Dense Optical Flow can be used to measure the strengths and direction of neural flow and how Helmholtz Decomposition can help to distinguish it from neural sources and areas of extended recurrent activity. As not only features but characteristic patterns are of interest it was further assessed how Autoencoders can be used for dimensionality reduction of the measured properties.\\
In the following the results of named literature review are presented first (section \ref{Chapter2}). It shows that anaesthetics change important properties of neurons most prominently their excitability which explains why bistable states of network activity can be observed in neocortex empirically (see section \ref{effects_of_anaesthesia}). While important differences exist between anaesthesia and sleep, slow waves can be observed in both states. One may distinguish corticothalamic slow waves in the delta range and neocortical slow waves of a duration above one second (see section \ref{slow_waves_anaesthesia_sleep}). The latter kind of slow waves is of main interest here. Neocortical slow waves are understood as extended periods of recurrent neural activation that manifests as above baseline activity (see section \ref{working_definition}). \\
The methods to analyse slow waves are explained in the second part. First a new approach is presented that allows to track haemoglobin rich blood in vessels of various size. It is shown that high frequency oscillations result from intracranial blood flow and band-stop filtering can be used to reduce this effect. Second an overview of the processing steps is provided. The procedure of splitting events is reported, then Dense Optical Flow is introduced, and it is demonstrated how Helmholtz Decomposition can be used to distinguish between the effect of local brightness changes and global flow. Subsequently the use of Autoencoders for unsupervised characterization of slow waves is discussed and the hyperparameters of the models used are presented. \\
The results of the approach are shown in the third part. First simple statistics regarding the dataset, selected features and the experimental conditions are reported. Then the relationship between properties of waves is summarized for representative events. Finally, the results of the Autoencoders are reported. \\
In the last part the findings regarding the methods and the achieved results are discussed. Optical Flow, Helmholtz decomposition and Autoencoders are a suitable tool to study slow waves. Feature engineering helps to select relevant and robust features. Optical Flow can be used in combination with Helmholtz-Decomposition to distinguish between local sources and global pathways of neural flow. Autoencoders represent an easy-to-use tool for the analysis of high-level features. The results confirm the existence of fundamentally different types of slow waves. The approach allows for a fine-grained distinctions and reveals a complex topology of static and dynamic properties in latent slow wave space.\\
