% Chapter Template

\chapter{Slow waves} % Main chapter title

\label{Chapter2} % Change X to a consecutive number; for referencing this chapter elsewhere, use \ref{ChapterX}
\label{review} % For referencing the chapter elsewhere, use \ref{Chapter1}


%----------------------------------------------------------------------------------------
%	SECTION 1
%----------------------------------------------------------------------------------------

\section{Overview of types, functions and modes of action}
\label{overview_of_functions_and_types}
There is a large body of literature that addresses slow oscillations in the brain. Slow waves in the delta range (0.5-4 Hz) can been identified in electroencephalograms (EEG) in deep sleep and under anaesthesia. Neurons that switch between up and down states in the respective frequency are present in thalamus indicating that this kind of slow waves is of thalamic origin \parencite[1110]{brown2012control}. As they interact with cortex they are reffered to as thalamocortical slow waves here. Since their discovery by Steriade et. al \parencite*{steriade1993novel} a different type of slow waves that occur with a frequency below 1Hz have been studied extensively as well: Neocortical slow oscillations. As they do not necessarily occur in regular time intervals they are referred to as neocortical slow waves here (see also section \ref{working_definition}). Both electrophysiological recordings as well as two-photon and wide field fluorescence microscopy have been used to investigate neocortical slow waves \parencite{niethard2018cortical, celotto2020analysis}. Neocortical slow waves with a typical duration of more than one second are the main subject of the empirical analysis presented here.\\
 On one side neocortical and thalamocortical slow waves are recognized as distinct phenomena \parencite[p. 1110]{brown2012control}. On the other side it was found that both signals relate can occur in an orchestrated way. One interpretation is that neocortical slow oscillations bind together spindles and delta waves \parencite[p. 1110]{brown2012control}. While named signals are of different origin one could hence consider them to be the signature of a single distributed process. To better understand how distinct oscillatory patterns in the brain relate to neocortical slow waves is crucial to summarize our knowledge about them. Hence selected examples for literature on slow waves, sleep and the effect of anaesthetics are briefly discussed in this chapter. Herein the aim is not to provide a full literature review but to present relevant information that is necessary to discern events, measure their properties and interpretate the results achieved \footnote{For a complete review on neocortical slow waves see \parencite{neske2016slow}.}.\\
Anaesthesia with isoflourane has several effects on the properties and the spiking behavior of neurons. The modes of action of anaesthetics such as isoflourane have been studied extensively \parencite{qazzaz2017modulation, moghadam2019comparative, eger1981isoflurane, jenkins1999effects}. Although the precise mechanisms are not fully understood it shows that anaesthetics change several properties of neurons and reduce their excitability which explains widespread quiesence at deep levels of anaesthesia. At intermediate levels bistable states can be observed in which slow waves of distinct shapes occur (see section \ref{effects_of_anaesthesia}).\\
The question how bistable states arise from the interaction of cells in neural networks has been addressed using spiking network models. They provide an explanation for the occurrence of neocortical slow waves during deep sleep and under anaesthesia. The circumstance that several models exist highlights not only that different explanations are possible but also that different kinds of processes might exist \parencite{nghiem2018two}. Bistable states occur both during sleep and during anaesthesia. However important characteristics of the dynamics of neocortical slow waves differ between these states. Hence it was argued that two different types of slow waves occur during sleep and anaesthesia (see section \ref{mechanistic_models}). \\
Evidence from several decades of research highlights the role that thalamic circuits play for the generation of rhythmical activity, the awake state and slow wave sleep \parencite{brown2012control}. In contrast to anaesthesia sleep is a vegetative state. The decrease in excitation that can be explained by the interaction of anaesthetics and membrane proteins is arguably achieved by processes that include the suppression of excitatory signals from the ascending reticular activation system  (ARAS) during deep sleep. Thalamic nuclei play a crucial role in the generation of neural oscillations including spindles and slow waves in the delta range. While slow wave sleep is arguably promoted by several mechanisms including the cardian cycle and the release of melatonin that alters neural excitability, sleep spindles of thalamic origin are assumed to be the trigger for a transition to deep sleep \parencite[p. 347]{montagna2005fatal}. Besides, specific regions have been identified in thalamus that oscillate in the delta range \parencite{steriade1984thalamus}. Sleep spindles also precede many of the neocortical slow waves that were observed on the level of single cells using two-photon imaging \parencite{niethard2018cortical}. This contrasts with the assumption that neocortical slow waves form in a spontaneous and fully independant manner in neocortex if the excitability is reduced (see section \ref{slow_waves_anaesthesia_sleep}).\\
A putative function of neocortical slow waves is memory consolidation. The assumed modes of action have been described in the hippocampo neocortical dialog model \parencite{buzsaki1996hippocampo}. Many results support the hypothesis that memory replay occurs during sleep and that it includes both hippocampus and neocortex. One may hypothesize that neocortical slow waves represent the neural signature of an important part of this process. Memory consoldation is arguably the most prominent cognitive function of slow waves (see section \ref{slow_waves_and_memory}).\\
As explained above slow waves differ with respect to several features. This holds both for the duration and frequency at which they occur and the assumed sites of origin. Neocortical slow waves can emerge from various regions. EEG studies in humans indicate that prefrontal-orbitofrontal regions act as the preferred source location \parencite[p. 1110]{brown2012control}. First steps to distinguish slow waves in a data driven approach were carried out by Bernardi et al. \parencite*{bernardi2018local}. Two types of sleep slow waves may be distinguished in EEG. This distinction is based on a metric that includes a shape parameter and a term that indicates whether slow waves are rather local or widespread. The distibution of these types of slow waves differs during different stages of deep sleep. It highlights the importance of means to discern slow waves based on their shape and spatial pattern of activation. A working definition that allows to detect and separate slow waves is derived from the developed understanding of slow waves (see section \ref{working_definition}).

%----------------------------------------------------------------------------------------
%	SECTION 2
%----------------------------------------------------------------------------------------

\section{Effects of isoflourane}
\label{effects_of_anaesthesia}
Anaesthetics alter the spiking behaviour of cortical neurons. Under deep anaesthesia the activity of cells is highly reduced and quiescence can be observed for most units. However, the modes of action that cause the respective changes of neural properties differ between anaesthetic agents. In addition, the effects can depend on the exact concentration of the drug in use and low dosages may, in some cases, even have opposing effects \parencite{moghadam2019comparative}. As isoflourane was used to aquire the dataset at hand its effects on individual cells is shortly summarized highlighting important differences to other anaesthetics when necessary. It is known that alterations of the properties of individual cells change the dynamics of neural interaction on a population level. By adjusting the properties of cells in simple models of neural networks accordingly bistable states can be reproduced that resemble slow waves under anaesthesia \parencite{jercog2017up}.\\
The exact pharmacological mechanisms of action of inhalant anaesthetics such as isoflourane remain uncertain \parencite{miller2020inhalational}. Effects on several ion channels have been reported including both chemically gated CL- channels (GABA receptors and glycine receptors) and K+ channels (Glutamate receptors). For example isoflourane is known to reduce the hyperpotentiation that results from CL- influx as a consequence of in vitro GABA administration \parencite{jenkins1999effects}. Hence it can be assumed to have excitatory effects in the nervous system as it decreases the inhibition due to GABA. However isoflourane also has inhibitory effects as it supresses K+ channel currents resulting in smaller electrically triggered peak amplitudes of action potentials \parencite{buljubasic1992effects}. Besides a potentiation of glycine receptors is assumed alongside other neurochemical mechanisms that alter the excitabilty of neurons \parencite{pubchem2020iso}. In vivo studies indicate that the net-effect of isoflourane appears to be inhibitory for all relevant dosages. In this respect isoflourane contrasts with other anaesthetics (including e.g. halothane and ketamine) that show concentration dependance \parencite{moghadam2019comparative}. It shall be noted however that differences in the density of different types of receptors exist in different areas of the brain. While isoflourane can be assumed to decrease the excitability of neurons and inhibit neural signaling at all concentrations this effect could differ between brain areas.\\
Single cell recordings reveal decisive effects of isoflourane on the properties and the spontaneous bahavior of neurons. Moghadam et. al \parencite*{moghadam2019comparative} performed a comparative study of systemic and volatile general anesthetics in single cell cultures and the isolated brain of lymnaea stagnalis. The substances tested include the volatile agents sodium pentobarbital, sodium thiopentone, ketamine on one side as well as halothane, enflurane and isoflurane on the other. It was found that isoflourane causes a gradual decline of both amplitude and frequency of spontaneous action potentials. Differences between the six types of neurons studied were found to be marginal. Upon stimulation neurons remained silent at all examined concentrations of isoflourane. In contrast a gradual decline of evoked action potentials showed for inceasing levels of enflourane. Using either halothane or barbiturates the authors were also able to produce bistable states in vitro during which periods of rapid spiking and quiesence alternate spontaneously.\\
Besides named alternations in spiking, effects on subthreshold properties of neurons have been identified. Increasing levels of isofourane can cause a decrease in the membrane time constant i.e. the duration between stumulus onset and 63\% potential change of the cell membrane. Under normal conditions it takes longer for the neuron to reach maximal voltage as compared to anaesthesia with isoflourane. This is reflected in the estimated membrane capacitance that analogously decreases. The membrane capacitance is especially interesting because of the role it plays in the integration of electrical inputs \parencite{golowasch2014}. The most likely explanation for the apperent reduction is however an increase in the leakage current. This is because it is (1) rather unplausible for the capcitance of the membrane to change significantly as its surface area does not change and (2) because of the abovementioned interaction between membrane proteins and anaesthetics \parencite{qazzaz2017modulation}. Neurons act as temporal integrators and fire if the combined voltage of input spikes that occur simultanously or in short succession exceeds threshold. If the leakage current is stronger the timing of input spikes becomes more critical as the membrane potential goes back to baseline more quickly which may prohibit charge accumulation. Hence it can be hypothesized that anaesthetics could affect the temporal integration of signals.  \footnote{ Note that other anaesthetics were found to increase the membrane capacitance}\\
While named effects explain the inhibition of neural firing under anaesthesia for different amounts of isoflourane decisive effects arise on the population level. Eger \parencite*{eger1981isoflurane} systematically studied the EEG patterns in the awake state and during anaesthesia with isoflourane for five different dosages at up to 2.9\% in humans. For very light anaesthesia (iso = .56\%) low voltage fast activity can be observed. At a light surgical level (isp = .96\% and 1.78\%) slow oscillations are present that change from more regular to irregular patterns and alternating patterns with high amplitude oscillations at a moderate surgical level (iso = 2.2\%). For deep anaesthesia only occasional low voltage activity shows (iso = 2.9\%). Isoflourane administration leads to a gradual shift from a stable awake state to bistable states and finally deep anaesthesia where quiesence dominates.\\
It shows that inhalant anaesthetics interact with ion channels in the neural cell membrane. The overall effect is a reduced excitability that causes quiescence at high dosages. In contrast intermediate concentrations lead to a gradual change from a steady up state to more or less regular slow waves that occur sporadically at deep anaesthesia. Slow oscillations can be observed in the brain at intermediate levels of anaesthesia.

\section{Mechanistic models}
\label{mechanistic_models}
The effects of anesthesia that are presented here provide the background for a mechanistic interpretation of slow wave activity. The dynamics that arise from changes of neural properties can be modeled using spiking artificial neural networks. Several networks have been proposed.\\
For example, features of slow waves can be produced with a simple model that employs Adaptive Exponential Integrate And Fire cells as shown by Nghiem et al. \parencite{nghiem2018two}: Simulations of a network that consists of 80\% excitatory cells and 20\% fast spiking inhibitory neurons produce sequences of continuous activity (up-states) and widespread absence of action potentials (down-states). A shift from down-states to and up state can be triggered by background noise while "spike-frequency adaptation on excitatory cells produces a self-inhibition that, destabilizing the up state, causes a reset to the down state" (Nghiem et. al 2018, p. 2). Spiking network models can reproduce alternating sequences of up states and down states, an important characteristic of population level activity under anaesthesia. \\
A model that explains sleep slow waves on the basis of ion channels is the averaged neuron model. Accordingly slow wave occur because of an increase in Ca2+ concentration in the cell that can trigger Ca2+ gated K+ channels. During the slow wave up state intracellular Ca2+ is assumed to activate concentration dependant K+ channels which result in a membrane hyperpolarization due to K+ efflux. This hyperpolarization prohibits action potentials and hence leads to the transition to a down state. "These results suggested that activity-dependent K+ channels, such as KCa, might be crucial for the termination of Up states" \parencite{neske2016slow}. synaptic depression.???? It is hypothesized that "The bursting phase of the SWS firing pattern is initiated by Ca2+ entry mainly through the NMDA receptor (NMDAR) and voltage‐gated Ca2+ channels" (https://onlinelibrary.wiley.com/doi/full/10.1002/bies.201700105). According to the model incoming signals from glutaminergic neurons are hence able to trigger up states by activation of NMDA receptors wheares a transition to the down state is explained by accumulation of Ca2+ in the cells. \parencite{shi2019genes}.\\
While simple mechanistic models explain important features of anaesthesia on a population level they rely require the formalization of processes that are in their entirety not fully understood. It was highlighted that the exact mechanisms of action of anaesthetics remain uncertain. Generalizing over different agents and dosages is not necessarily justified and the dynamic of neural signalling might be affected by differences in the distribution of receptors for different neurotransmitters. Besides the complex anatomy of the brain including the various pathways that connect different regions uni- or bidirectionally is typically not reflected in simple mechanistic models. More precise measurements of the pathways and dynamics of neural signal transduction during slow wave anaesthesia are necessary to understand bistable states of the brain. This is especially important also because it was recently shown that slow wave sleep differs significantly from anaesthesia.

%-----------------------------------
%	SECTION 3
%-----------------------------------
\section{Sleep slow waves and related signals}
\label{slow_waves_anaesthesia_sleep}
Recently it was argued that the dynamics of slow wave sleep and slow wave anaesthesia differ substantially \parencite{nghiem2018two}. Jercog et al. \parencite*{jercog2017up} found that the length of up states and the subsequent down state is correlated during urethane induced anaesthesia in rats for clearly synchronized periods where high-amplitude, slow fluctuations are present in local field potentials. Coefficients indicate a very weak relationship (r = .2). However, the correlation was found to be consistently positive across experiments whereas the correlation with later down-state periods (time-lag > 2) is close to zero \parencite{jercog2017up}. This pattern in the dynamic of neural firing can be reproduced using simple mechanistic models (see section \ref{effects_of_anaesthesia }). During human non-REM sleep, however, no such relationship holds. As a tendency, long down states are followed by short up states instead as indicated by a very weak but significant negative correlation (r = -.04). A negative peak in the temporal cross correlation exists around zero \parencite{nghiem2018two}. This gave rise to the argument that sleep-slow waves are fundamentally different from their counterpart during anaesthesia \parencite{nghiem2018two}. Hence the mechanisms that enable slow wave sleep are more closely examined in this section. \\

%-----------------------------------
%	SECTION 3.1
%-----------------------------------

\subsection{Three systems for sleep regulation}
It is long known that the anatomical substrate of sleep is a distributed system rather than a single region in the brain\parencite{akert1965anatomical}. Mammalian sleep represents a physiological process that is arguably regulated by the interaction between various control mechanisms. In general, one may categorize these processes into three classes each of which affects the activity of cortical neurons directly or via intermediate events.
\begin{enumerate}[label={(\arabic*)}]
    \item First there are processes that decrease neural excitability rather unpacifically by a release or an accumulation of disperse chemical messengers such as the neurohormone melatonin.
    \item Second processes exist that incorporate electrical signalling via action potentials. They include the neurotransmitter systems that correspond to the ascending reticular arousal system (ARAS). Ascending projections from other areas (e.g. thalamic nuclei) that increase the neocortical arousal in a spatially rather unspecifically manner can be subsumed under this category as well.
    \item Third there are processes that lead to oscillations in the firing rate of distinct neural populations. The latter can include spike rates that follow the circadian rhythm and alter the both the release of neurohormones and the excitation of cortex via spiking activity. However oscillatory activity also occurs in the range of delta slow waves and below.
\end{enumerate}
As it is not possible to discuss all features of sleep here, important examples are presented that fall under named categories. The aim is to characterize the circumstances under which sleep slow waves occur \footnote{For an good review with a focus on sleep that includes a discussion about the origins of different kinds of rhythms in the brain see Brown et al. \parencite*{brown2012control}.}.\\
Melatonin is the messenger of a system for sleep regulation that falls under the first category. In healthy subjects, melatonin concentrations alternate according to the cardiac cycle \parencite{montagna2005fatal}. Melatonin can pass the blood brain barrier and can hence diffuse to the central nervous system where it accumulates \parencite{aulinas2019physiology}. It acts as a neurohormone and has an inhibitory effect on the excitability of cortical neurons via different modes of action. Melatonin receptors in the neural membrane have been identified. Besides it was found that melatonin interacts with voltage-sensitive Ca2+ channel and inhibits the release of neurotransmitters as well as synaptic transmission \parencite{choi2014melatonin}. As it decreases the neural excitability of cortical neurons it shares some of the effects of mild anaesthesia. Melatonin reduces the excitability of disperse populations of neurons in the central nervous system promoting drowsiness and sleep. \\
The suprachiasmatic nuclei (SCN) represent an example for the third of the abovementioned categories. It interacts with the melatonin system. Melatonin is produced by pinealocytes that compose 95\% of the cells in the pineal gland \parencite{aulinas2019physiology}. The pineal gland resides outside the blood brain barrier and extends hypothalamus ventrally and represents an interface between neural signalling end the endocrine system. It is innervated by centres in the brain stem. Most importantly it is known to receive signals which are related to retinal activity from the SCN. The SCN is capable not only of creating a light-dependant circadian rhythm, but also of maintaining an entrained rhythm \parencite{koella1984organization}. The latter could be demonstrated in electrophysiological recordings in vitro. Isolated SCN show firing rates that alternate in a 24 hour pattern \parencite{de2011melatonergic}. Arguably this explains changes of melatonin release that follows the circadian cycle.\\
While melatonin is evidently part of sleep regulation the transition between the awake state and slow wave sleep is arguably initiated by other mechanisms. Mammalian sleep can be triggered by electrical stimulation of the thalamus. First experiments were carried out using cats while later experiments confirmed that the effect exists for dogs as well \parencite{akert1951sleep, akimoto1956sleep}. A stimulation of the intralaminar thalamus of cats with a duration of 30-60 seconds and pulses at 4-8Hz leads to a transition to sleep. This transition occurs in different stages. Sleep spindles occur while the animal is still awake. After several minutes a full transition to slow wave sleep can be observed \parencite{akert1951sleep}. These early studies indicate that thalamocortical sleep spindles play an important role for the transition to sleep.\\
Early lesion studies indicated that brainstem nuclei are required for both normal awake activity and REM sleep. In animals with a Cerveau isolé preparation no normal sleep patterns can be observed. Instead one may synchronized firing that manifests in slow waves with a main frequency of around 1 Hz \parencite{kawamura1968hippocampal}. In named preparation ascending fibers between pons and midbrain are dissected. Effectively large parts of the reticular formation and several nuclei that use specific neurotransmitters are disconnected from midbrain, thalamus and cortex. In contrast an encéphale isolé preparation in which the brain and the spine are separated does not have named effects but normal sleep patterns can be observed. This highlights the importance of brain stem for the regulation of sleep and alertness. Named nuclei include serotoninergic neurons in the raphe nucleus, norepinephrine neurons in locus coeruleus and glutamate neurons in the pedunculopontine nucleus. They represent the most important excitatory neurotransmitters. These nuclei are known as the source of the ascending reticular activation system that is known to promote alertness \parencite{brown2012control}.\\
\\Similar to target sites in thalamus electrical stimulation of the raphe nuclei induces sedation and sleep in rats \parencite{kostowski1969electrical}. The raphe nuclei are brain stem areas that represent the origins of the ascending serotonergic pathways. Together with several neighboring cores they provide excitatory input to large populations of cortical neurons and hence faciliate firing via different pathways. If the activity of the raphe nuclei is reduced by electrical stimulation the arousal of neocortex arguably consequently decreases (see \ref{ascending_system_and_oscillators}). Mechanisms that lead to a transition to sleep modulate the activity of ascending pathways of the ARAS including the serotonergic system which increase cortical activation.\\
It shows that the mechanisms that regulate sleep decrease the excitability and activation of cortical neurons under the contribution of chemical messengers including melatonin and an inhibition of the ascending arousal system. Subcortical nuclei can produce oscillatory patterns that reflect the circadian cycle whereas thalamus plays a key role in the transition from an awake state to deep sleep. Deep sleep represents a bistable and is characterized by a switch between up an down states of cortical activity. Thalamic nuclei are known to produce sleep spindles which are considered a switch for the transition between light an deep sleep \parencite{montagna2005fatal}. Besides sleep spindles, thalamic regions generate rhythmical activity that oscillates in the delta range. Neocortical slow waves that arise during sleep arguably interact with these rythms. The underlying processes are more closely examined in the next section.

%-----------------------------------
%	SECTION 3.2
%-----------------------------------

\subsection{Neural oscillators and relays for ascending activation}
\label{ascending_system_and_oscillators}
Both the transition to the bistable state of slow wave sleep and fluctuations in the delta range include ascending signals that originate from subcortical  areas. Several centres have been identified that generate oscillatory patterns of spiking activity. These patterns include (1) thalamocortical sleep spindles, (2) hippocampal short wave ripples (3) delta waves of thalamic origin and (4) neocortical slow waves. Distinct pacemakers exist which neural oscillators and cause named rhythms in the brain.\\

\textbf{Rostral midline thalamus: A relay for ascending activation}\\
The circumstance that slow waves were present in the Cerveau isolé preparation indicates that slow waves are initiated in brain regions above pons. Thalamus is considered a relay for ascending signals in the awake state. It shows that certain nuclei that act as neural oscillators especially under certain conditions, most notably sleep. "During these states, the behavior of thalamic cells is characterized by long-lasting hyperpolarizations and phasic burst discharges recurring rhythmically" \parencite[p. 21]{steriade1984thalamus}. Chirugical removal of thalamus in cats was found to lead to a decrease of the amount of non-REM sleep from 38\% to 11.8\%. In contrast diencephalic cats still showed sleep spindles \parencite{montagna2005fatal}. Further support comes for example from fatal familial insomnia a deadly neurodegenerative desease. Findings indicate that thalamic atrophy causes a lack of sleep spindles and delta sleep which implies that "thalamus is the origin of slow wave sleep" \parencite[p. 339]{montagna2005fatal}.\\
While thalamic nuclei contain neurons that oscillate in the rythms of slow wave sleep and it is undoubtedly essential for the regulation of sleep it is unclear if delta oscillations are generated by distinct areas of thalamus. Evidence suggests, however, that Rostral midline thalamus (RMT) acts as a relay for ascending arousal signals that originate from the reticular formation and abolish slow waves. Stimulation of the midbrain reticular formation in cats suppresses slow thalamic rhythms of hyperpolarizing episodes \parencite{steriade1984thalamus}. RMT is part of the nonspecific thalamic system. It contains nuclei that project to cortex in a spatially unspecific manner. While specific roles of some cores can be attributed one can assume RMT to alter the excitabilty of cortical neurons in a spatially rather unspecific way. \parencite{vertes2015limbic}. Phasic desynchronization of the neocortex could only be observed upon stimulation of the midline thalamic area in cerveau isole preparations of cats \parencite{kawamura1968hippocampal}. Rostral midline thalamus relays signals of the ARAS that abolish slow waves.\\

\textbf{Lateral posterior nucleus: Source of delta oscillations}
While it was previously found that thalamic delta waves are absent in diencephalic cats where cortex is surgically disconnected from lower stuctures \parencite{villablanca2004counterpointing}, Dossi et. al \parencite*{dossi1992electrophysiology} identified thalamocortical in the posterior lateral posterior nucleus of thalamus (LPN) that show delta oscillations (here: 0.5 - 4 Hz) even after disconnection from related cortical areas. Intracellular recordings show highly rhythmical oscillations in membrane potential at around 0.5 Hz. These oscillations occur spontaneously. When exceeding a certain threshold each of these oscillations was found to generate exactely one action potential at peak deporalization. Moreover is showed that self-sustained delta oscillations can be elicited by electrical stimulation in short rhythmic pulses \parencite{dossi1992electrophysiology}. These findings (1) clearly that LPN is a thalamicortical center that generates delta waves and (2) that rhythmical excitation entrains neurons to exhibit a delta rythm themselves.\\
 More recently the T-type calcium channels have been found to play a key role in thalamocortical circuits that generate slow waves in the delta range. EEG delta waves are practically absent during NREM sleep in knockout mice that lack the alpha G1 subunit of named calcium channel. Moreover the duration of NREM sleep is generally reduced \parencite{lee2004lack}. However selective modification of T-type calcium channels in RMT were found to have similar effects \parencite{brown2012control}. This potentially indicates an abolishing effect due to changes in the thalamic sites of the ARAS.\\
As thalamic cores and cortex are strongly connected it was argued that they represent a common functional unit for the generation of delta waves. The circumstance that rhythmical stimulation elicits delta waves indicates that interactions between oscillations in thalamus and cortex exist. Potentially they bind activity in thalamocortical networks. However the oscillators that produce slow waves in the delta range rely in thalamus.\\

\textbf{Thalamic reticular nucleus: Generator for sleep spindles}\\
 Another thalamic nucleus is considered the source of sleep spindles. Sleep spindles manifest as patterns of damped oscillations in EEG. Sleep spindles are well visible in the frequency band between 7 and 15 Hz \parencite{niethard2018cortical}. These patterns in EEG arguably arise from synchronous firing of cortical pyramidal cells. It is known however that sleep spindles originate from the thalamic reticular nucleus \parencite{luthi2014sleep}. Thalamocortical sleep spindles are considered to be a switch for the transition between light an deep sleep \parencite{montagna2005fatal}. In many cases they coocur with activity of cortical cells measured by two photon flourescence microscopy \parencite{niethard2018cortical}. The reticular thalamic nucleus is considered the pacemaker for sleep spindles that interact with slow waves.\\
 Spindle activity occurs not only during sleep but also under anaesthesia with barbiturates.\\

\textbf{Hippocampal formation: Origin of sharp wave ripples}\\
Another area in the brain that is known to act as a neural oscillator resides in the hippocampal formation. Hippocamus represents the oldest part of cortex and exibits a highly structured nature that has been extensively studied. An important feature of the organization of hippocampus is the overall flow of signals that follows the scheme of the tri synaptic way. Signals from entorhinal cortex travel in circular patterns through hippocampus. Activation propagates from entorhinal cortex to the dentate gyrus, subsequently to the CA3 subfield, via the Schaffer collaterals further to subfield CA1 and finally back to the entorhinal cortex. This network topology is incorporated in spiking neural networks that can reproduce important features of hippocampal sharp wave ripples \parencite{aussel2018detailed}.\\
Hippocampal sharp wave ripples can be produced in vitro. Long term potentiation of neurons that form the recurrent networks that include subfield CA3 is known to facilitate the generation of hippocampal sharp wave ripples . Repeated high- frequency stimulation of a particular area in subfield CA1 induces long term potentiation and enables short wave ripples both in area CA3 and CA1. These findings do not only indicate the source of origin of hippocampal sharp wave ripples but indicate also its involvement in processes of memory formation \parencite{behrens2005induction}.\\

\textbf{Neocortex: Source of neocortical slow waves}\\
Neocortical slow waves occur spontaneously during deep sleep and deep anaesthesia in irregular rhythms with a frequency of one Hz and below. Differences have been found for ketamine induced anaesthesia and urethane where frequencies were found to be slighly lower \parencite{steriade1993novel}. Neocortical slow waves have been described as reoccuring bursts of action potentials of neurons in several regions of cortex. Simultaneous aqusition of electrocorticograms (ECOG) shows that these bursts coocur with an oscillation in the electrical potential on the cortical surface \parencite{steriade1993novel}. Arguably these oscillations relate to the neocortical slow waves [< 1Hz] that can be measured using EEG during deep sleep. Simultaneous ECoG and widefield flouroscence imaging shows that up states typically cooccur with several oscillations of the ECoG potential (Stroh et al. \cite*{stroh2013making}; see figure 2.1). \\
Simultaneous recording of ECOG and local field potentials (LFP) in the form of a electrothalamigram indicates the presence of a corresponding thalamic signal \parencite{steriade1993novel}. The hypothesis that neocortex contains networks that act as a neural oscillator has been confirmed both empirically and by means of simulation (see section \ref{effects_of_anaesthesia}). However it was also found that optical stiumuli and the resulting signal that orinates from the retina are suitable to elicit slow waves. Besides optogenetic stimulation of the lateral geniculate body, the thalamic relay for signals from the retina, causes slow waves. Neocortical slow waves can hence not only arise because of random random noise that manifests as spontaneous firing but can also be triggered by perceptual inputs and excitatory projections from subcortical sites. Moreover it was shown that slow waves evoked in cortex strictly preceede a signal that can be measured in thalamus. \parencite{stroh2013making}. Taken together this indictes that cortex is the site of origin for this type of slow waves. Oscillations arguably arise spontaneously while the interaction with other signals can determine the precise time of the occurance of neocortical slow waves.\\
Two photon flouroscence microscopy reveals a differential behavior of different neurons during up states of slow wave anaesthesia. It shows that neurons do not all behave in the same way but peak activity cooccurs with sleep spindles only for some neurons. In average the flourescence signal correlates with slow waves measures using ECoG. As mentioned above the electrophysiological response does however not directly reflect the mean signal. Different neurons have specific response properties and coocur with other oscillations in the brain in a selective manner (see also \ref{slow_waves_and_memory}).\\


\textbf{Interaction effects}\\
Interaction effects between signals that arise from thalamus and those in cortex have been identified. Thalamus is evidently the source of sleep spindles and delta waves, the most important patterns of slow wave sleep. However it shall be mentioned that thalamus is strongly interconnected with cortex. Electrical stimulation of the reticular core of thalamus was found to cause an increased in amplitude of delta waves in hippocampus. In contrast rhythmical stimulation of cortical area 7 evokes periodic oscillations in thalamic neurons that sustain over extended periods of time after stimulation offset. The resulting thalamic rhythm can in return enhance synchronization \parencite[p. 21]{steriade1984thalamus}. This finding indicates not only that entrained rhythms exist in thalamus but also that oscillations in thalamus and both archiocortex and neocortex strongly relate to each other.\\
Sleep spindles, hippocampal sharp wave ripples and neocortical slow waves are arguably not fully independant. For example the spindle amplitude is correlated (r = .3) with the df/f signal in wake active cells. A weaker yet significant correaltion is found for wake inactive cells. The cooccurence of slow oscillations and sleep spindles is also correlated with effects on the calcium flourescence. A higher percentage change can be observed when both signals coincide as compared to the case of sleep spindles alone being present during a slow wave in calcium flourescence \parencite{niethard2018cortical}. Coningencies between hippocampal short wave ripples and slow waves and their role for memory consolidation are discussed in the next section.\\
Figure \ref{oscillatory_centres_and_the_ascending_reticular_activation_system} illustrates the location of oscillatory nuclei which evidently play a role in slow wave sleep and shows the ventral and dorsal pathway of the ascending reticular activation system. The SCN that arguably plays an important role in the regulation of sleep according to the circadian cycle is additionally indicated. Note that oscillatory centres in the rostral medulla that orchestrate breathing during sleep are not indicated \parencite{kubin2019interactions}.\\
While the named structures are arguably the source of the respective oscillation it must be noted that signals interact. They do not occur fully independantly. Up and down states of neocortical slow waves arise in cortex while they can be caused by subcortical afferences. Empirical data indicates that oscillations in archiocortex and thalamus are more or less tightly bound to neocortical slow waves \parencite{niethard2018cortical}. The presumed role of cortico-hippocampal synchrony for memory consollidation is discussed in the next section.\\

\begin{figure}[th]
\centering
\includegraphics[width=\textwidth,height=\textheight,keepaspectratio]{Figures/oscillatory_centres_and_the_ascending_reticular_activation_system}
\decoRule
\caption[Oscillatory centres and the ascending reticular activation system]{Oscillatory centres and the ascending reticular activation system.\\
Circular arrows illustrate sites that act as neural oscillators. Thalamocortical slow waves in the delta range are assumed to originate from the lateral posterior nucleus of thalamus. Thalamocortical sleep spindles are produced in the reticular thalamic nucleus. Rostral midline thalamus (RMT) arguably acts as a relay for the ARAS and abolishes delta waves. It projects to cortex in a non-specific manner. Hippocampal sharp-wave-ripples arise in the subfield CA3 of hippocampus proper. Slow cortical oscillations (<1Hz) and high frequency waves arguably arise in neocortex. Straight arrows indicate the lateral and dorsal pathways of the ascending reticular activation system. Note that centres in the basal forebrain (BF) may suppress the generation of hippocampal short-wave-ripples. Arrows do not relate to the anatomical location of projection fibres and indicate no preferential sites within the target structure. The transections of the preparations “ceveau isole” ad “encephale isole” are indicated by dotted lines. Slow waves occur in the cerveau isole while normal sleep patterns have been observed for encephala isole. Abbreviations: Reticular thalamic nucleus (RTN), Lateral posterios nucleus (LP) Medial nucleus of the preoptic area MnPO and ventrolateral nucleus of the preoptic area (VLPO),  Basal forebrain including the
Medial septum (MS) and the Vertical Limb of the Diagonal Band (VDB), suprachismatic nucleus (SCN), rostral midline thalamus (RMT), Nuclei of the ascending reticular activation system (ARAS nuclei) including serotoninergic neurons in the raphe nucleus, norepinephrine neurons in locus coeruleus and glutamate neurons in the pedunculopontine nucleus. While oscillatory centres are isolated inter projections allow for interaction that can cause patterns to occur at the same time. Own depiction inspired by \parencite[1099]{brown2012control}.}
\label{fig:oscillatory_centres_and_the_ascending_reticular_activation_system}
\end{figure}

%-----------------------------------
%	Section 4
%-----------------------------------

\section{Sleep slow waves and memory consolidation}
\label{slow_waves_and_memory}

Sleep slow waves arguably represent an important mechanism for memory consolidation. According to the hippocampal-neocortical dialogue model of slow waves, the interaction of hippocampal sharp wave ripples and neocortical slow waves fosters memory consolidation (Buzsáki 1989, Walker et al. 2009): Recently it was shown that generating neocortical slow waves in prefrontal networks such that they are coupled to the occurrence of short wave ripples in the hippocampus increases the performance of rats in a memory task (Maingret 2018). Coherently, a correlation of slow wave activity and the brain-derived neurotrophic factor was found for humans (Duncan, 2013). Spontaneously occurring slow waves arguably play a similar role for memory. They predominantly occur phase locked to sleep spindles (Demanuele et. al. 2017). However, the exact mechanisms that orchestrate this synchrony are unknown (Sanda et. al 2020).\\
The hippocampo neocortical dialog model... Entorhinal cortex is the target structure of the Van Essen diagram. Visual information can be assumed to be processed in a hierarchial manner along the ventral pathway. Wheares response properties in early visual cortices relate to physical features of the visual percept such as simple shapes that occur in a specific receptive field neurons evidently code for more abstract concepts in higher areas. This indicates that hippocampus receives a highly processed and argaubly relatively abstract, sparse code.\\
Hippocampus can be understood as an autoassociative memory. Unique structure. Schaffer collatorals. Long term potentiation occurs. However long term memory encoded in neocortex. Movements can be elicited in motor cortex wheares sensory perceptions can be triggered by stimulation of somatosensory areas. Higher motor cortices code for more complex actions. The question the hippocampo-neocortico dialogue model aims to answer is how experiences that are encoded in hippocampus are consolidated by the interaction with neocortex. Slow waves arguably play an important role in this respect.\\
Evidence suggests that the synchrony between slow waves and hippocampal sharp wave ripples during sleep enhances learning. This was demonstrated in a stimulation paradigm where slow waves are triggered such that they coincide with sleep spindles measured in the hippocampal formation (Buzsáki 1989, Walker et al. 2009). ??More information necessary here??. It is however not fully understood why increasing the synchrony between two signals that can oscillate independantly forsters memory consolidation.\\
Empirical data shows that memory replay occurs both in hippocampus and the associated cortical manifestations of the previously encoded learning experience. Ji and Wilson \parencite*{ji2007coordinated} investigated contingencies between behavior and slow wave sleep to study the interaction between hippocampus and neocortex in the aquistion of long term memories. They studied the response properties of neurons in rats in an experimental paradigm that includes maze running in an eight shaped maze. Place cells were identified that fire for different regions of the maze in ?? where ???. Depending on the direction of running different patterns in the neural response. The sequence of peak activities of the recorded neurons codes for the direction in which the rat passes through the maze. It was found that these sequences are repeated in hippocampus during slow wave sleep. Most interestingly, however, memory replay did not only take place in archiocortex but signals appeared to be exported to cortex. The same sequences of neural responses were measurable in cortex ??where exactely?? \parencite{ji2007coordinated}.\\
 As mentioned above source modeling of sleep slow waves indicates cingulate fiber trajectories. Moreover, it was found that at least two different types of slow waves exist that potentially relate to distinct synchronization processes (Bernardi 2018). Because of its high spatial resolution fluorescence microscopy can provide more fine-grained distinctions and may potentially reveal trajectories of neural signal transduction during slow wave anesthesia. This highlights the importance of methods that allow to capture the variance of temporospatial patterns of neural slow waves.

%-----------------------------------
%	Section 5
%-----------------------------------

\section{Neocortical slow waves: Towards a working definiton}
\label{working_definition}
A distinction between slower and shorter waves has already been made in the pioneering electroencephalographic study almost a century ago \parencite[p. 550]{berger1929}. Shortly later the term delta wave was coined to describe oscillations in the recorded voltage in the respective frequency band. Delta waves are defined for a frequency of 1 to 4 Hz by some authors \parencite{kubin2019interactions} while others define the delta range for oscillations between 0.5 to 4Hz \parencite{dossi1992electrophysiology}. The term slow wave is also used to describe oscillatory patterns in EEGs in the delta range. "Slow-wave activity (SWA) is defined as the EEG power in the slow wave frequency band" \parencite[p. 1]{furrer2019sleep}. Hence the term delta wave is sometimes used as a synonym for slow waves. However, when speaking of slow waves it is oftenly referred to slow wave sleep or anaesthesia where delta waves occur largely independant from higher frequency components in EEG.\\
Depending on the context slow waves relate to a pattern in EEG or oscillations that can be measured in a methodologically rather agnostic way. For example changes in the spike rates of neurons are also refferred to as slow waves \parencite{jercog2017up}. This shows that the term slow wave is used in relation to the neural phenomenom, its functions and its signatures that can be measured using various methods.\\
Both delta waves and neocortical slow waves represent slow waves in the wide sense. For example Celotto et. al \parencite*{celotto2020analysis} denote neocortical slow oscillations as slow waves. Furrer et al. \parencite*{furrer2019sleep} use the term as a synonym for delta waves instead. As explained in the previous sections both phenomena can occur independantly from one another. It is hence advisable to use distinct terms. Steriade et. al \parencite*{steriade1993novel} suggested the label neocortical slow oscillation. It was noted, however, that the up states and down stated do not necessarily appear in a rythmical pattern \parencite{brown2012control}. In addition the transition to an up state appears as a slowly travelling wave. As the term is more descriptive this phenomenon is hence referred to as neocortical slow wave as opposed to thalamocortical slow waves here. Sleep slow waves show more irregular patterns in EEG and could potentially include signatures of both neocortical slow waves and thalamocortical slow waves in the delta range \parencite{steriade1993novel}.\\
Slow waves are understood as short periods of above baseline activity that occur in bistable states such as deep sleep and anaesthesia which are characterized by reduced neural excitabilty. Slow waves reflect the up states that are interlaced with down states of neural activity \parencite{jercog2017up}. They are dynamic patterns of recurrent acticity of large potentially disctinct populations of neurons. Slow waves can differ with respect to the pathway of neural flow especially in early stages and with respect to the set of recruited cells where recurrent activity prevails. They appear as propagating waves in widefield and two-photon calcium imaging in vivo \parencite{celotto2020analysis, niethard2018cortical}. This arguably reflects a runaway process: If activity exceeds the threshold neurons fire and cause postsyaptic potentials which can elicit action potentials in the target neurons \parencite{nghiem2018two}. Because of the spatial arrangement of neurons in cortex where close sites are preferabbly connected more strongly one can arguably observe travelling waves. Traveling waves can even be observed in neocortical slices in vitro \parencite{wu2008propagating}. Cortex is known to be of a highly recurrent nature \parencite{guamuanuct2017mouse}. Arguably this is why one can observe extended periods of activation as these neurons can be assumed to directly or indirectly project back to the one that initially fired. Hence, slow waves can reveal regions of high functional connectivity. However also cingulate fiber trajectories were identified in EEG during sleep \parencite{murphy2009source}. Neocortical slow waves could also travel through cortex via the highly structured fiber bundles of cingulum that connects entorhinal cortex and the bordaring hippocampus with with neocortical sites.\\
 If hippocampal short wave rippels correlate with cortical up-states one may speak of a hippocampo neocortical slow wave event. Analogously one could form subcategorizations for slow waves that incorporate deviant sets of connected neurons. One may hypothesize that neocortical slow waves incorporate signals that do not only show functionally connected populations of neuons but pathways of dynamical flow.\\
Here data from widefield flouroscence imaging is analyzed. This raises the question how slow oscillations in calcium imaging differs from their counterpart in EEG or ECoG. The simultaneous aquisition of ECoG and wide field GCamp imaging reveals how both signals relate to each other. Slow waves in the mean flourescence signal cooccur with slow waves in the electrophysiological recording \parencite{stroh2013making}. The respective correlation could also be demonstrated on the basis of individual cells using two photon imaging \parencite{niethard2018cortical}. It shall be noted however that Spontaneously occuring neocortical slow waves differ with respect to the frequency and the
\\
As slow waves are understood as short periods of above baseline activity they can incorporate multiple oscillations. These subsegments could potentially relate to distinct neural events. The circumstance that slow waves can include several oscillations is especially important for the detection and separation of events. It means, however, that events could not be separated based on local minima alone \parencite{celotto2020analysis}. However, it is reasonable to assume that multi peak slow waves exist. Conceptualizing slow waves as above baseline activity bears potentials for revealing the temporal dynamics of slow waves that incorporate several stages. It also means that new methods are required for the analysis the dynamic of slow waves.

\begin{figure}[th]
\centering
\includegraphics[width=\textwidth,height=\textheight,keepaspectratio]{Figures/ecog_calcium_flourescence}
\decoRule
\caption[Frequencies of slow waves differ between measuring techniques]{Frequencies of slow waves differ between measuring techniques.\\
Data was recovered from supplementary figure 4A, Stroh et al. \parencite*{stroh2013making} for secondary analysis. Signals are smoothed using a gaussian filter and subsequently normalized. Baseline correction was applied for the hemodynamic calcium signal (lower left). The resulting noise reduced signals were investigated with respect to the frequencies present. Upper row: The normalized Electrocorticogram (left) and the frequencies power revealed by fourier transform (right). Lower row: The normalized signal change of the hemodynamic response (left) and the corresponding frequency spectrogram (right). The frequency power is plotted in blue whereas a smoothed version is plotted in orange. It shows that the peak frequency of the hemodynamic signals relies below 0.2 Hz whereas the peak frequency of the calcium signal is above 0.7. Note also that slow waves of both signals are aligned but several ECoG oscillations occur during an hemodynamic event.\\
}
\label{fig:ecog_calcium_flourescence}
\end{figure}
