% Chapter Template

\chapter{Slow-waves} % Main chapter title

\label{Chapter2} % Change X to a consecutive number; for referencing this chapter elsewhere, use \ref{ChapterX}

%----------------------------------------------------------------------------------------
%	SECTION 1
%----------------------------------------------------------------------------------------

\section{Overview of proposed types and functions}
There is a large body of literature that addresses slow oscillations in the brain. Slow waves in the delta range (~1-4 Hz) can been identified in electroencephalograms (EEGs). How evern more recently slow-waves have also been studied using wide-field flourescence microscopy. Simultaneous imaging and electrophysiological recordings indicate how both signals relate to each other. As a new approach to analyze the resulting data is presented here it is crucial to summarize our knowledge about slow-waves. A clear understanding is not only necessary to discern events and measure slow-wave properties but it is also important for the cross-validation and interpretation of the results achieved. Hence relevant literature on slow-waves, sleep and the effect of anaesthetics is briefly reviewed in this chapter. \\
The effects of anaesthesia on the properties and the spiking behavior of individual cells has been studied extensively. It shows that anaesthetics change several properties of neurons which modify the temporal and spatial integration of signals and cause bistable states in distributed neural networks. This can be demonstrated using spiking network models which provide an explanation for the occurence of slow-waves under anaesthesia that is coherent with named changes of neural properties (see ???). Stronger inhibition of neural activity also exists during sleep. However important characteristics of slow-wave dynamics differ between NREM sleep and anaesthesia. Hence it was argued that two different types of slow-waves occur during sleep and anaesthesia. \\
Evidence from several decades of research highlights the role thalamic circuits play for the generation of rhythmical activity and sleep. While slow-wave sleep is arguably promoted by several mechanisms including the cardian cycle that alters neural excitabilty, findings suggest that deep sleep is triggered by sleep spindles of thalamic origin. This contrasts with with the idea of slow-waves that form spontaneously in different regions of cortex if the excitabilty is reduced. Not all mechanisms of sleep are fully understood. It must be mentioned, however, that sleep is a vegetative state and thalamic nuclei play a crucial role in the generation of neural oscillations including slow-waves.\\
Sleep slow-waves are known to be involved in processes of memory consolidation. The assumed modes of action have been described in the hippocampal neocortical dialog model. Many results support the hypothesis that memory replay occurs during sleep and that it includes both hippocampus and neocortex. One may assume that sleep-slow waves represent the neural signature of an important part of this process. Memory consoldation is arguably the most prominent function of slow-waves (see ??).\\
First steps to distinguish slow-waves in a data driven approach were carried out by Bernardi et al. (2018). Two types of sleep slow-waves may be distinguished in EEG. This distinction is based on a metric that includes a shape parameter and a term that indicates whether slow-waves are rather local or widespread. The distibution of these types of slow-waves differs during different stages of deep sleep. It highlights the importance of means to discern slow-waves based on their shape and spatial pattern of activation.\\
A working definition that allows to detect and separate slow-waves is derived from the developed understanding of slow-waves. 

\section{Effects of anaesthesia on single cells and neural networks}
Anaesthetics alter the spiking behavior of neurons. Under deep anaesthesia the activity of cells is highly reduced and quiescence can be observed for most units. However, the modes of action that cause the respective changes of neural properties differ between anaesthetic agents. In addition, the effects can depend on the exact concentration of the drug in use and low dosages may, in some cases, even have opposing effects. As isoflourane was used to aquire the dataset at hand its effects on individual cells is shortly summarized highlighting important differences to other anaesthetics when necessary. It is known that alterations of the properties of individual cells change the dynamics of neural interaction on a population level. By adjusting the properties of cells in simple models of neural networks bistable states can be reproduced that resemble slow-waves under anaesthesia.\\
The exact pharmacological mechanisms of action of inhalant anaesthetics such as isoflourane remain uncertain (Miller et al. 2020). Effects on several ion channels have been reported including both chemically gated CL- channels (GABA receptors and glycine receptors) and K+ channels (Glutamate receptors). For example isoflourane is known to reduce the hyperpotentiation that results from CL- influx as a consequence of in vitro GABA administration (Jenkins 1999). Hence it can be assumed to have exitatory effects  in the nervous system as it decreases the inhibition due to GABA. However isoflourane also has inhibitory effects as it supresses K+ channel currents resulting in smaller electrically triggered peak amplitudes of action potentials (Buljubasic 1992). Besides a potentiation of glycine receptors is assumed alongside other neurochemical mechanisms that alter the excitabilty of neurons (National Center for Biotechnology 2021). In vivo studies indicate that the net-effect of isoflourane appears to be inhibitory for all relevant dosages. In this respect isoflourane contrasts with many other anaesthetics (including e.g. halothane and ketamine) that show concentration dependance (Mogdahan 2019). It shall be noted however that differences in the density of different types of receptors may exist  in different brain regions. While isoflourane can be assumed to decrease the excitability of neurons and inhibit neural signaling at all concentrations the strengths of decrease could differ between brain areas.\\
Single cell recordings reveal decisive effects of Isoflourane on the properties and the spontaneous bahavior of neurons. Moghadam et. al (2019) performed a comparative study of systemic and volatile general anesthetics in single cell cultures and the isolated brain of lymnaea stagnalis. The substances tested include the volatile agents sodium pentobarbital, sodium thiopentone, ketamine on one side as well as halothane, enflurane and isoflurane on the other. It was found that isoflourane causes a gradual decline of both amplitude and frequency of spontaneous action potentials. Differences between the six types of neurons studies were found to be marginal. Upon stimulation neurons remained silent at all studied concentrations of isoflourane. In contrast a gradual decline of evoked action potentials showed for inceasing levels of enflourane. Using either halothane or barbiturates the authors were also able to produce bistable states in vitro during which periods of rapid spiking and quiesence alternate spontaneously.\\
Besides named alternations in spiking, effects on subthreshold properties of neurons have been identified. Increasing levels of isofourane can cause a decrease in the membrane time constant i.e. the duration between stumulus onset and 63\% potential change of the cell membrane. Under normal conditions it takes longer for the neuron to reach maximal voltage as compared to anaesthesia with isoflourane. This is reflected in the estimated membrane capacitance that analogously decreases. The membrane capacitance is especially interesting because of the role it plays in the integration of electrical inputs (Golowasch \& Nadim 2014). The most likely explanation for the apperent reduction is however an increase in the leakage current. This is because it is (1) rather unplausible for the capcitance of the constant size membrane to change significantly and (2) because of the abovementioned interaction between membrane proteins and anaesthetics (Quazzaz \& Winlow 2017). Neurons act as temporal integrators and fire if the combined voltage of input spikes that occur simultanously or in short succession exceeds threshold. If the leakage current is stronger the timing of input spikes becomes more critical as the membrane potential goes back to baseline more quickly which may pohibit charge accumulation. Hence it can be hypothsized that anaesthetics affect temporal integration.  \footnote{ Note that other anaesthetics were found to increase the membrane capacitance}\\
Arguably the abovementioned effects explain important features of population level dynamics. Marshall (1981) systematically studied the EEG patterns in the awake state and during anaesthesia with isoflourane for five different dosages at up to 2.9\% in humans. For very light anaesthesia (iso = .56\%) low voltage fast activity can be observed. At a light surgical level (isp = .56\% and 1.78\%) slow oscillations are present that change from more regular to irregular patterns and alternating patterns with high amplitude oscillations at a moderate surgical level (iso = 2.2\%). For deep anaesthesia only occasional low voltage activity shows (iso = 2.9\%). leads to a gradual shift from stable up state to bistable states with distinct patterns and finally deep anaesthesia where quiesence dominates.\\
The effects of anesthesia presented here provide the background for a mechanistic interpretation of slow-wave activity. The overall dynamics that arise from changes of neural properties can be modeled using a simple model that employs Adaptive Exponential Integrate And Fire cells as shown by Nghiem et al. (2018): Simulations of a network that consists of 80\% excitatory cells and 20\% fast spiking inhibitory neurons produce sequences of continuous activity (up-states) and widespread absence of action potentials (down-states). A shift from down-states to and up state can be triggered by background noise while "spike-frequency adaptation on excitatory cells produces a self-inhibition that, destabilizing the up state, causes a reset to the down state" (Nghiem et. al 2018, p. 2). Spiking network models can reproduce alternating sequences of up states and down states, an important characteristic of population level activity under anaesthesia. \\
While simple mechanistic models explain some of the features of anaesthesia on a population level they represent a coarse generalization. It was highlighted that the exact mechanisms of action of anaesthetics remain uncertain. Generalizing over different agents and dosages is not necessarily justified and the dynamic of neural signalling might be affected by differences in the distribution of receptors for different neurotransmitters. Besides the complex anatomy of the brain including the various pathways that connect different regions uni- or bidirectionally is typically not reflected in simple mechanistic models. More precise measurements of the pathways and dynamics of neural signal transduction during slow-wave anaesthesia are necessary to understand bistable states of the brain. This is especially important also because it was argued that slow-wave sleep differs significantly from anaesthesia.
%-----------------------------------
%	SUBSECTION 1
%-----------------------------------
\section{Two types of slow-waves in anaesthesia and deep sleep}

Recently it was argued that the dynamics of slow-wave sleep and slow-wave anaesthesia differ substantially (Ngiehm et. al 2018). Jercog et al. (2017) found that the length of up states and the subsequent down state is correlated during urethane induced anaesthesia in rats for clearly synchronized periods where high-amplitude, slow fluctuations are present in local field potentials. Coefficients indicate a very weak relationship (r = .2). However, the correlation was found to be consistantly positive across experiments whearas the correlation with later down-state periods (time-lag > 2) is close to zero (Jercog et al. 2017). This pattern in the dynamic of neural firing can be reproduced using simple mechanistic models (see 2.2). During human non-REM sleep, however, no such relationship holds. As a tendancy, long down states are followed by short up states instead as indicated by a very weak but significant negative correlation (r = -.04). A negative peak in the temporal corsscorrelation exists around zero (Ngiehm et. al 2018). This gave rise to the argument that sleep-slow waves are fundamentally different from their counterpart during anaesthesia (Ngiehm et. al 2018). It allows for the interpretation that sleep slow-waves are elicited by a process that is not explained by spontaneous firing due to random cortical noise.\\
Sleep can be described as a physiological process that includes two seperate effects. First a decrease in neural excitability due to changes in the concentrations of disperse neurotransmitters and neurohormones and second a processes that repeatatively toggles down and up-states.\\
Serotonin and melatonin. Pineal gland. The vast majority of cells are pinealocytes (95\%) that produce melatonin (Aulinas 2019). Inhibitory. It was found that melatonin interacts with voltage-sensitive Ca2+ channel and inhibits the release of neurotransmitters as well as synaptic transmission (Choi 2014). in vitro. The scn projects to the pineal gland. The SCN is assumed to be capable not only of creating a light-dependant circadian rhythm, but also of maintaining an entrained rhythm (Koella 1984).  
\\
Sleep can be triggered by electrical stimulation of thalamus. Cats dogs. Delayed.
Electrostimulation of the reticular core was also found to be followed by an increased theta output in the hippocampus. In this respect spindles. They are considered a switch for the transition between light an deep sleep(Montagna 2005).
Rats. Electrical stimulation of the raphe nuclei induces sedation and sleep in rats. Raphe nuclei are the area of origin of the ascending serotonergic pathways.\\

Thalamus gateway to consciousness. In awake. Relay. However it was found that the thalamus can also act as neural oscillator. In sleep. Also under anaesthesia with barbiturates. "During these states, the behavior of thalamic cells is characterized by long-lasting hyperpolarizations and phasic burst discharges recurring rhythmically" (Steriade 1984, p. 21). Thalamic -> Athalamic cat: non-REM sleep 38\% -> 11.8\% of total observation time. Diencephalic cats (ablation of the entire neocortex and striatum): Still Sleep spindles (after Montagna 2005).\\
•	lesioning the reticular nucleus also decreased delta waves and led the animals to sudden death.36 \\
 Findings from fatal familial insomnia indicate congruently that thalamic atrophy is "associated with lack of sleep spindles and delta sleep implicate the thalamus in the origin of slow wave sleep"(Montagna 2005). Preferential thalamo-olivary degeneration. "paramedian thalamus acts as ‘final common pathway’"??\\

Rhythmical stimulation of cortex evokes periodic oscillations in thalamic neurons that sustain over extended periods of time after stimulation offset.  cortical area 7. Once in a bistable state it can self-enhance synchronization (Steriade 1984).\\

sleep as an instinctive behaviour.	medial thalamo-limbic structures as mechanistically involved in SWS production and autonomic balance


that include the cardian cycle sleep spindles are seen as a crucial signal that triggers a between up and down-states. 

This gives rise to the question: What is sleep and how is it different from anaesthesia? what is sleep.
Explains dynamics during anaesthesia not sleep. Sleep.... hippocampus.
Long up states --> long down states. during anaesthesia. 

Functions of sleep. short. Ending with the role of memory formation. Question wheather it is different is unclear.
The circumstance that it is debated whether sleep-slow waves and slow-waves under anaesthesia are comparable highlights the importance of an approach that allows to automatically characterize these patterns of neural signals. 


Methodological triangulation.


%-----------------------------------
%	SUBSECTION 2
%-----------------------------------

\section{Sleep slow-waves and memory consolidation}
According to the hippocampal-neocortical dialogue model of slow-waves, the interaction of hippocampal sharp wave ripples and cortical slow-waves fosters memory consolidation (Buzsáki 1989, Walker et al. 2009): Recently it was shown that generating cortical slow-waves in prefrontal networks such that they are coupled to the occurrence of short wave ripples in the hippocampus increases the performance of rats in a memory task (Maingret 2018). Coherently, a correlation of slow-wave activity and the brain-derived neurotrophic factor was found for humans (Duncan, 2013). Spontaneously occurring slow-waves arguably play a similar role for memory. They predominantly occur phase locked to sleep spindles (Demanuele et. al. 2017). However, the exact mechanisms that orchestrate this synchrony are unknown (Sanda et. al 2020). Moreover, it was found that at least two different types of slow-waves exist that potentially relate to distinct synchronization processes (Bernardi 2018). Because of its high spatial resolution fluorescence microscopy can provide more fine-grained distinctions and may potentially reveal trajectories of neural signal transduction during slow wave anesthesia. This highlights the importance of methods that allow to capture the variance of temporospatial patterns of neural slow-waves.


\section{A working definiton}
A distinction between slower and shorter waves has already been made in the pioneering electroencephalographic study almost a century ago \parencite[p. 550]{berger1929}. Shortly later the term delta wave was coined to describe oscillations in the recorded voltage in the respective frequency band of around 1 to 4 Hz. The term slow-wave is typically also used to describe oscillatory patterns in EEGs. "Slow-wave activity (SWA) is defined as the EEG power in the slow wave frequency band" \parencite[p. 1]{furrer2019sleep}. This frequency band relates to the delta band. Hence the term delta wave which is oftenly used in EEG research can be considered a synonym for slow waves. However, when speaking of slow-waves one oftenly referrs to slow wave sleep or anaesthesia where delta waves occur largely independant from higher frequency components.\\
As explained above slow waves are however more then an oscillation in the scalp potential relative to a reference electrode. The term is also used in relation to the neural phenomenom, its functions and its signatures that can be measured using various methods. Slow-waves are short periods of above baseline activity that occur in bistable states such as deep sleep and anaesthesia which are characterized by reduced neural excitabilty. Slow-waves are dynamic patterns of recurrent acticity of large potentially disctinct populations of neurons. They can differ with respect to the pathway of neural flow and the set of recruited cells. Sleep slow waves represent an important mechanism for memory consolidation. Evidence suggests that memory replay occurs both for sparse hippocampal codes and the associated dense cortical manifestations of the previously encoded learning experience. If thalamic short wave rippels correlate with cortical up-states one may speak of a thalamocortical slow wave event. Analogously one could form subcategorizations for slow-waves that incorporate deviant sets of temporally connected neurons. One may hypothesize that slow waves show neural associations in temporal isolation.  \\
If slow-waves are understood as short periods of above baseline activity they can hence incorporate multiple peaks or slow oscillations in the strict sense. This is especially important for the detection and separation of events. Instead of relying on local minima alone this definition requires to detect in addition wheather activity went back to baseline. While conceptualizing slow waves as slow oscillations in the strict sense this approach chosen. It bears potentials of revealing the temporal dynamics of slow waves that incorporate several stages.